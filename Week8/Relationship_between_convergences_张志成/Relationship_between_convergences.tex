\documentclass[UTF8, 12pt]{article}
\fontfamily{DENGL.TTF}
\usepackage[a4paper, total={6in, 8in}]{geometry}
\usepackage{xeCJK}
\usepackage{enumitem}
\usepackage{amsmath}
\usepackage{mathtools}
\usepackage{amssymb}
\usepackage{amsfonts}
\usepackage{mathrsfs}
\usepackage{XCharter}
\usepackage{fancyhdr}
\usepackage{eulervm}
\usepackage{graphicx}
\usepackage{mdframed}
\usepackage{ntheorem}

\pagestyle{fancy}

\newenvironment{proof}{\noindent\ignorespaces\textbf{Proof:}}{\hfill $\square$\par\noindent}
\newenvironment{solution}{\noindent\ignorespaces\textbf{Solution:}}{\hfill $\square$\par\noindent}

\newtheorem{claim}{Claim}
\newtheorem{problem}{Problem}
\newtheorem{lemma}{Lemma}
\newtheorem{theorem}{Theorem}
\newtheorem*{remark*}{Remark}

\title{Relationships between Convergences}
\author{张志成 518030910439}
\date{\today}

\begin{document}
    \maketitle

    \begin{theorem}
        Almost sure convergence $\Longrightarrow$ Convergence in Probability. \\
        Let $\mu$ be a measure $\Sigma \mapsto [0, \infty)$ and $(f_n) \in m\Sigma$, $f \in m\Sigma$.  \\
        If $$ f_{n} \overset{a.s.}{\longrightarrow} f ,$$ 
        i.e. $$ \mu(f_{n} \not\rightarrow f) = \mu(\{\omega\in S \ |\ f_{n}(\omega) \not\rightarrow f(\omega)\}) = 0, $$ 
        then for any $\epsilon > 0$, $$ \mu(|f_n-f| > \epsilon) \to 0 .$$
    \end{theorem}

    \begin{proof}
        \begin{align*}
            (f_{n} \not\rightarrow f) &= \{\omega\in S \ |\ f_{n}(\omega) \not\rightarrow f(\omega)\} \\
            &= \bigcup_{k=1}^{\infty}\bigcap_{m}\bigcup_{n = m}^{\infty}\ \{\omega \in S \ |\ |f_n(\omega) - f(\omega)| > \frac{1}{k} \} \\
            &= \bigcup_{k=1}^{\infty}\bigcap_{m}\bigcup_{n = m}^{\infty}\ (|f_n- f| > \frac{1}{k}).
        \end{align*}
        For any $\epsilon > 0$,
        since $\mu(f_{n} \not\rightarrow f) = 0$,
        we can deduce that $$ \mu(\ \bigcap_{m}\bigcup_{n = m}^{\infty}\ |f_n- f| > \epsilon) = 0 ,$$ thus
        \begin{align*}
            \lim_{n\to\infty}\mu(|f_n-f| > \epsilon) &\leq \lim_{m\to\infty}\mu(\bigcup_{n = m}^{\infty}  |f_n-f| > \epsilon) \\
            &= \mu(\lim_{m\to\infty}\bigcup_{n = m}^{\infty}  |f_n-f| > \epsilon)\quad \text{(MON of Measure)}\\
            &= \mu(\bigcap_{m}\bigcup_{n = m}^{\infty}  |f_n-f| > \epsilon) \\
            &= 0.
        \end{align*}
        Therefore $$ \lim_{n\to\infty}\mu(|f_n-f| > \epsilon) = 0 ,$$ i.e. $(f_n)$ converges to $f$ in probability.
    \end{proof}

    \begin{theorem}
        Convergence in Probability $\Longrightarrow$ Almost sure convergence of a subsequence. \\
        Let $(f_n) \in m\Sigma$, $f \in m\Sigma$. If for any $\epsilon > 0$, $$ \mu(|f_n-f| > \epsilon) \to 0 ,$$ \\
        then there exists an increasing subsequence $(n_k)$ such that $$ f_{n_k} \overset{a.s.}{\longrightarrow} f ,$$ 
        i.e. $$ \mu(f_{n_k} \not\rightarrow f) = \mu(\{\omega\in S \ |\ f_{n_k}(\omega) \not\rightarrow f(\omega)\}) = 0. $$ 
        % or equivalently,$$ \mu(\{\omega\ |\ f_{n_k}(\omega) \not\rightarrow f(\omega)\}) = 0 $$
    \end{theorem}

    \text{}

    \begin{proof}
        As a premise, we have $$ \mu(|f_n-f| > \epsilon) \to 0. $$ \\
        For an arbitrarily fixed $\epsilon > 0$, we will have an increasing sequence $(n_k)$ such that
        $$ \mu(|f_{n_k} - f| > \epsilon) < \frac{1}{2^k}.$$ \\
        Taking the infinite sum on both sides yields, $$ \sum_{k=1}^{\infty} \mu(|f_{n_k} - f| > \epsilon) < \sum_{k=1}^{\infty} \frac{1}{2^k} = 1 < \infty .$$ \\
        By the \textit{First Borel-Cantelli Lemma on general measure space}, we have
        \begin{align*}
            \mu(|f_{n_k} - f| > \epsilon, i.o.) &= 0.
            % \Longleftrightarrow \mu(|f_{n_k} - f| \leq \epsilon, ev) &= 1.
        \end{align*}
        Based on the arbitrariness of $\epsilon$, we have proven the statement, $$ \mu(f_{n_k} \not\rightarrow f) = 0 .$$
    \end{proof}

    \begin{remark*}
        The almost sure convergence might not hold for the whole sequence. \\
        The construction I resorted to is relatively complicated. Let $(S,\Sigma, \mu) = ([0,1], Leb[0,1], Leb)$. \\
        Define $f_n$ and $f$ as following: For every $x \in [0,1]$, 
        \begin{align*}
            f_1(x) = x &+ I_{[0,1]} \cdot x \\
            f_2(x) = x + I_{[0,\frac{1}{2}]} \cdot x ,&~ f_3(x) = x + I_{[\frac{1}{2}, 1]} \cdot x \\
            f_4(x) = x + I_{[0,\frac{1}{3}]} \cdot x, ~ f_5(x) = x &+ I_{[\frac{1}{3}, \frac{2}{3}]} \cdot x, ~ f_6(x) = x + I_{[\frac{2}{3}, 1]} \cdot x,
        \end{align*}
        and $$ f(x) = x. $$
        Then we have $$ \mu(|f_n-f| > \epsilon) < \frac{1}{\sqrt{2n} + 1} \to 0, $$
        however $$ \mu(f_n \not\rightarrow f) = Leb([0,1]) \neq 0. $$
        Therefore, given convergence in probability, only a subsequence of $(f_n)$ converges almost surely.
    \end{remark*}
    
    \text{}

    \begin{theorem}
        Convergence in Probability of Monotone $(f_n)$ $\Longrightarrow$ Almost sure convergence. \\
        We say $(f_n)$ is \textbf{monotone} in the sense that $$ \forall \omega \in S\ f_{n+1}(\omega) > f_n(\omega) .$$
        Let $(f_n)$ be a monotone sequence of measurable functions. \\ If for any $\epsilon > 0$, $$ \mu(|f_n-f| > \epsilon) \to 0 ,$$ \\
        then $$ f_{n} \overset{a.s.}{\longrightarrow} f ,$$ 
        i.e. $$ \mu(f_{n} \not\rightarrow f) = \mu(\{\omega\in S \ |\ f_{n}(\omega) \not\rightarrow f(\omega)\}) = 0 .$$ 
    \end{theorem}

    \text{}

    \begin{proof}
        Using the result of \textit{Problem 2}, we can obtain a subsequence $(n_k)$ such that  $$ \mu(f_{n_k} \not\rightarrow f) = \mu(\{\omega\in S \ |\ f_{n_k}(\omega) \not\rightarrow f(\omega)\}) = 0 .$$ \\
        Consider any $\omega$ such that $f_{n_k}(\omega) \rightarrow f(\omega)$. \\
        We prove that with monotonicity of $(f_n)$, $f_{n}(\omega) \rightarrow f(\omega)$. \\
        \begin{align*}
            f_{n_k}(\omega) \rightarrow f(\omega) \Longrightarrow &\forall \epsilon > 0\ \exists K\ \forall k > K\ |f_{n_k}(\omega) - f(\omega)| < \epsilon \\
            \xRightarrow{\text{Monotone}} &\forall \epsilon > 0\ \exists K\ \forall n > n_K\  |f_{n}(\omega) - f(\omega)| < \epsilon \\
            \Longrightarrow &f_{n}(\omega) \rightarrow f(\omega) 
        \end{align*}
        Therefore, $$ \mu(f_{n} \not\rightarrow f) \leq \mu(f_{n_k} \not\rightarrow f) = 0, $$
        thus $$ f_{n} \overset{a.s.}{\longrightarrow} f .$$ 
    \end{proof}
\end{document}