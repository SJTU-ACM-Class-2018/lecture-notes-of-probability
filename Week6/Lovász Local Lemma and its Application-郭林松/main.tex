\documentclass{article}
\usepackage[utf8]{inputenc}
\usepackage{amssymb}
\usepackage{amsmath}
\usepackage{amsthm}
\usepackage{geometry}
\usepackage{booktabs}
\usepackage{bm}
\usepackage{tcolorbox}
\usepackage{setspace}

\renewcommand{\baselinestretch}{1.3}

\newtheorem{theorem}{Theorem}
\newtheorem{lemma}[theorem]{Lemma}
\newtheorem{claim}[theorem]{Claim}
\newtheorem{proposition}[theorem]{Proposition}
\newtheorem{corollary}[theorem]{Corollary}
\newtheorem{fact}[theorem]{Fact}
\newtheorem{definition}[theorem]{Definition}
\newtheorem{remark}[theorem]{Remark}
\newtheorem{question}[theorem]{Question}
\newtheorem{answer}[theorem]{Answer}
\newtheorem{exercise}[theorem]{Exercise}
\newtheorem{example}[theorem]{Example}
\newtheorem{observation}[theorem]{Observation}


\title{Lovász Local Lemma and its Application}
\author{Guo Linsong 518030910419}
\date{\today}

\begin{document}

\maketitle


\begin{tcolorbox}
    \begin{lemma}
        Let $(E_1,E_2,\cdots,E_N)$ be a sequence of events. For each $E_i$, we have $P(E_i)\leq p$ and $E_i$ depends on at most $d$ other events. If $4dp\leq 1$, then
        $$P\left(\bigcap_{i=1}^N E_i^c\right)\geq (1-2p)^N>0$$
   \end{lemma}
\end{tcolorbox}

\begin{tcolorbox}
\begin{claim}
    Let $S\subseteq \{1,2,\cdots,N\}$, then for every $i$, we have
    $$P\left(E_i\Bigg| \bigcap_{j\in S}E_j^c\right)\leq 2p$$
\end{claim}
\end{tcolorbox}

Assuming the claim, it's easy to prove the Lovász Local Lemma.

$$P\left(\bigcap_{i=1}^N E_i^c\right)=\Pi_{i=1}^NP\left(E_i^c\Big| \bigcap_{j<i} E_j^c\right)=\Pi_{i=1}^N\left(1-P\left(E_i \Big| \bigcap_{j<i} E_j^c\right)\right)
\geq \left(1-2p\right)^N>0$$

Next we proof the claim.

\begin{proof}
We prove the claim by induction on $|S|$. If $|S|= 0$, the claim holds, because
$$P\left(E_i\Bigg| \bigcap_{j\in S}E_j^c\right)=P\left(E_i\right)\leq p \leq 2p$$

Assume that the claim holds when $|S|<s$. We will prove the claim for $|S|=s$. Suppose that $D$ be the set of all j such that $E_i$ depends on $E_j$. Here are two cases.

\textbf{case 1.} $S \cap D = \emptyset$

$$P\left(E_i\Bigg| \bigcap_{j\in S}E_j^c\right)=P(E_i)\leq p \leq 2p$$

\textbf{case 2.} $S \cap D \neq \emptyset$

Let $E_D = \bigcap_{j\in D}E_j^c$ and $E_{S\setminus D}=\bigcap_{j\in S\setminus D}E_j^c$
\begin{equation}\label{eq1}
    \begin{array}{rl}
        P\left(E_i\Big| \cap_{j\in S}E_j^c\right) 
        = &  P\left(E_i\Big| \left(E_D\bigcap E_{S\setminus D}\right)\right) \\
        = &  \frac{P\left(E_i \bigcap E_D\bigcap E_{S\setminus D}\right)}
        {P \left(E_D\bigcap E_{S\setminus D}\right)} \\
        = &\frac{P\left(E_i \bigcap E_D\big| E_{S\setminus D}\right)P\left(E_{S\setminus D}\right)}
        {P\left(E_D\big| E_{S\setminus D}\right)P\left(E_{S\setminus D}\right)} \\
        = &\frac{P\left(E_i \bigcap E_D\big| E_{S\setminus D}\right)}
        {P\left(E_D\big| E_{S\setminus D}\right)} \\
    \end{array}    
\end{equation}

For the numerator, we have
\begin{equation}\label{eq2}
P\left(E_i \bigcap E_D\Big| E_{S\setminus D}\right) \leq P\left(E_i\Big| E_{S\setminus D}\right)=P\left(E_i\right)\leq p    
\end{equation}

Note that $S \cap D \neq \emptyset$, so $|S\setminus D|<|S|$. We can apply the inductive hypothesis, so for all $E_k\in E_D$, $P(E_k|E_{S\setminus D})\leq 2p$.  For the  denominator, we have
\begin{equation}\label{eq3}
   \begin{array}{rl}
        P\left(E_D| E_{S\setminus D}\right)
        = & P\left(\bigcap_{k\in D}E_k^c \Big| E_{S\setminus D}\right) \\
        = & P\left(\left(\bigcup_{k\in D}E_k\right)^c\Big| E_{S\setminus D}\right) \\
        = & 1-P\left(\bigcup_{k\in D}E_k\Big| E_{S\setminus D}\right) \\
      \geq & 1-\sum\limits_{E_k\in E_D}P\left(E_k\Big| E_{S\setminus D}\right) \\
      \geq & 1-d\times 2p \\
      \geq & \frac{1}{2}  \\
   \end{array}
\end{equation}

Combining (2) and $(3)$ , we have $ P\left(E_i\mid \cap_{j\in S}E_j^c\right) = \frac{P\left(E_i \bigcap E_D\mid E_{S\setminus D}\right)}
        {P\left(E_D| E_{S\setminus D}\right)}\leq 2p$. We have proved the claim.
\end{proof}

\begin{tcolorbox}
    \begin{example}
       In a k-SAT formula like $(x_1 \vee \neg x_3 \vee x_5) \wedge (\neg x_2 \vee x_4) \wedge (x_3 \vee \neg x_4 \vee x_6)$, every variable $x_i=\{0, 1\}$ occurs in at most $\frac{2^k}{4k}$ different clauses. Then there exists an assignment satisfies all the clauses in the formula.
   \end{example}
\end{tcolorbox}

\begin{proof}
Let event $E_i$ be that the $i$-th clause is not satisfied. If the $i$-th clause has more than $k$ variables, we only need to care $k$ of them because of the feature of the operation $\vee$. Thus $P(E_i) \leq \frac{1}{2^k}$. Every variable occurs in at most $\frac{2^k}{4k}$ different clauses, so the $i$-th clause depends on at most other $(\frac{2^k}{4k}-1)\times k\leq \frac{2^k}{4}$ events. Note that $4\times\frac{2^k}{4}\times\frac{1}{2^k} = 1$, Lovász Local Lemma implies $P\left(\bigcap_{i=1}^N E_i^c\right)>0$, which means there exists an assignment satisfies all the clauses in the formula.
\end{proof}

\begin{tcolorbox}
    \begin{example}
       Let $G=(V,E)$ be a graph. For every vertex $v$, its color is $C_v=\{1,2,\cdots,k\}$ and its degree is at most $m$. If $m\leq \frac{k}{8}$, then there must exist an assignment of $\{C_v|v\in V\}$ such that every edge connects two vertexes of different colors.    
   \end{example}
\end{tcolorbox}

\begin{proof}
Let event $E_e(e=(u,v)\in E)$ be that $e$ connects two vertexes of the same color. Note that $P(E_e) = \frac{\sum_{C_u=1}^k\sum_{C_v=1}^k[C_u=C_v]}{k^2}=\frac{1}{k}$. Every vertex's degree is at most $m$, so the event $E_e$ depends on at most other $2(m -1)\leq 2m \leq \frac{k}{4}$ events. Node that $4\times\frac{k}{4}\times\frac{1}{k}=1$,  Lovász Local Lemma implies $P\left(\bigcap_{e\in E} E_e^c\right)>0$, which means there must exist an assignment  such that every edge connects two vertexes of different colors. 
\end{proof}
\end{document}
