% This is a template for lecture notes.
\documentclass{article}
\usepackage{cite}
% \usepackage[UTF8]{ctex}
\usepackage{amssymb}
\usepackage{amsmath}
\usepackage{amsthm}
\usepackage{geometry}
\usepackage{booktabs}
\usepackage{bm}
\usepackage{enumerate}
\usepackage{tcolorbox}
% \CTEXoptions[today=old]
%Some commonly used notations
%\geometry{a4paper,bottom = 3cm,left = 3cm, right = 3cm}

%for reference
\usepackage{hyperref}
\usepackage[capitalise]{cleveref}
\crefname{enumi}{}{}

\newtheorem{theorem}{Theorem}
\newtheorem{lemma}[theorem]{Lemma}
\newtheorem{proposition}[theorem]{Proposition}
\newtheorem{corollary}[theorem]{Corollary}
\newtheorem{fact}[theorem]{Fact}
\newtheorem{definition}[theorem]{Definition}
\newtheorem{remark}[theorem]{Remark}
\newtheorem{question}[theorem]{Question}
\newtheorem{answer}[theorem]{Answer}
\newtheorem{exercise}[theorem]{Exercise}
\newtheorem{example}[theorem]{Example}
%\newenvironment{proof}{\noindent \textbf{Proof:}}{$\Box$}
\newtheorem{observation}[theorem]{Observation}

%to use newcommand for convenience
\newcommand{\mbb}{\mathbb}
\newcommand{\mbf}{\mathbf}
\newcommand{\mbz}{\mathbb{Z}}
\newcommand{\mbn}{\mathbb{N}}
\newcommand{\mbp}{\mathbb{P}}
\newcommand{\mbh}{\mathbb{H}}
\newcommand{\mbq}{\mathbb{Q}}
\newcommand{\vep}{\varepsilon}
\newcommand{\rd}{\mathrm{d}}
\newcommand{\inv}{^{-1}}
\newcommand{\hp}{^\prime}
\newcommand{\mca}{\mathcal{A}}
\newcommand{\mcb}{\mathcal{B}}
\newcommand{\mcc}{\mathcal{C}}
\newcommand{\mcm}{\mathcal{M}}
\newcommand{\mcr}{\mathcal{R}}
\newcommand{\mcf}{\mathcal{F}}
\newcommand{\mfa}{\mathfrak{A}}
\newcommand{\mfb}{\mathfrak{B}}
\newcommand{\mfc}{\mathfrak{C}}
\newcommand{\mfi}{\mathfrak{I}}
\newcommand{\Iff}{\mbox{iff }}
\newcommand{\AND}{\mbox{ and }}

%this is how we define operators.
\DeclareMathOperator{\rank}{rank} % rank

\title{Notes on the Proof of Lovasz Local Lemma}
\author{Lu Jiaxin\\
Student ID: 518030910412}
\date{\today}

\begin{document}
    \maketitle
\begin{tcolorbox}
\begin{lemma}[\textbf{Lovasz Local Lemma}\cite{MR0382050}]\label{LLL}
For events $E_1, \ldots, E_n$ with a dependency graph $G$, suppose,
\begin{enumerate}[(1)]
\item $\forall i \exists p\in(0,1), P(E_i) \leq p$
\item $\max \deg_G(V) \leq d$
\item $4dp\leq 1$
\end{enumerate}
then,
\[
    P(\bigcap E_i^C) > 0
\]
\end{lemma}
\end{tcolorbox}

\begin{proof}
To prove the lemma, we need to first introduce two statements.

For $s=0,1,\ldots,N-1$, $\forall |S| \leq s$,
\begin{enumerate}[(a)]
    \item $$P(\bigcap_{j\in S}E_j^C) > 0$$
    \item $$\forall k\in[N]\backslash S, P(E_k \cap \bigcap_{j\in S}E_j^C) \leq 2p P(\bigcap_{j\in S}E_j^C)$$
\end{enumerate}

We first prove two statements by induction.

\textbf{$s = 0$ :} When $s = 0$, we have $S = \emptyset$, so,
\begin{enumerate}[(a)]
    \item \[
        P(\bigcap_{j\in S=\emptyset} E_j^C) = 1 > 0
    \]
    \item \[
        \frac{P(E_k \cap \bigcap_{j\in S}E_j^C)}{P(\bigcap_{j\in S}E_j^C)} = P(E_k) \leq p \leq 2p
    \]
\end{enumerate}

\textbf{$s > 0$ :} Suppose two statements holds for $0,\ldots,s-1$.

For $s$,

\begin{enumerate}[(a)]
    \item
    \begin{align}\label{eq1}
        P(\bigcap_{j\in S}E_j^C) = 
        \frac{P(\bigcap_{j\in[s]}E_j^C)}{P(\bigcap_{j\in[s-1]}E_j^c)} \times
        \frac{P(\bigcap_{j\in[s-1]}E_j^c)}{P(\bigcap_{j\in[s-2]}E_j^c)} \times
        \cdots \times
        \frac{P(\bigcap_{j\in[2]}E_j^c)}{P(\bigcap_{j\in[1]}E_j^c)}\times
        \frac{P(\bigcap_{j\in[1]}E_j^c)}{1}
    \end{align}
    Since statements (b) holds for $0, \ldots, s-1$, $\forall 1 \leq s'\leq s$,
    \[
        \frac{P(\bigcap_{j\in[s']} E_j^C)}{P(\bigcap_{j\in[s'-1]} E_j^C)} \geq 1-2p
    \]
    So,
    \[
        P(\bigcap_{j\in S}E_j^C) \geq (1-2p)^n
    \]
    Since we have $4dp\leq 1$, we can derive $2p \leq \frac{1}{2d} \leq \frac{1}{2}$, so, $1-2p\geq \frac{1}{2}$.
    
    Thus, \[
        P(\bigcap_{j\in S}E_j^C) > 0
    \]

    \item We need to prove \[
        \frac{P(E_k \cap \bigcap_{j\in S}E_j^C)}{P(\bigcap_{j\in S}E_j^C)} = P(E_k \mid \bigcap_{j\in S}E_j^C) \leq 2p
    \]

    Let $S_1 = {j\in S : j \sim k \mbox{ in } G}, S_2 = S \backslash S_1$.

    When $S_1 = \emptyset$, $P(E_k \mid \bigcap_{j\in S}E_j^C) = P(E_k) \leq p < 2p$.

    Otherwise, $S_1 \not = \emptyset, |S_2| < S$.
    
    Let $F_{S_1} := \bigcap_{j\in S_1}E_j^C, F_{S_2} := \bigcap_{j\in S_2} E_j^C$, we have,
    \begin{align*}
        P(E_k \cap F_{S_1} \mid F_{S_2}) &\leq P(E_k \mid F_{S_2}) \\
        &=  P(E_k) \leq p
    \end{align*}
    and,
    \begin{align*}
        P(F_{S_1} \mid F_{S_2}) &= P(\bigcap_{i\in S_1}E_i^C \mid \bigcap_{j\in S_2} E_j^C) \\
        &\geq 1 - \sum_{i\in S_1} P(E_i \mid \bigcap_{j \in S_2} E_j^C) \\
        &\geq 1-2pd \geq \frac{1}{2}
    \end{align*}

    So, 
    \begin{align*}
        P(E_k \mid \bigcap_{j\in S}E_j^C) &= P(E_k \mid F_{S_1} \cap F_{S_2}) \\
        &= \frac{P(E_k \cap F_{S_1} \mid F_{S_2})}{P(F_{S_1} \mid F_{S_2})} \\
        &\leq \frac{p}{\frac{1}{2}} \leq 2p
    \end{align*}
\end{enumerate}

Now we have both statements (a) and (b) stand, we substitute $N$ into $s$ in \cref{eq1}. We now can derive that all the denominators on the right hand side is larger than 0, and,
\[
    P(\bigcap E_i^C) = P(\bigcap_{i\in[N]}E_i^C) \geq (1-2p)^N \geq \left(\frac{1}{2}\right)^N > 0
\]

\end{proof}

Here is an application of \cref{LLL}.

\begin{tcolorbox}
    \begin{definition}[Conjunctive Normal Form]\label{CNF}
        A formula is said to be in Conjunctive Normal Form (CNF) if it consists of AND's of several clause. For instance, $(x\vee y)\wedge(y\vee \neg z \vee w)$ is a CNF formula.
    \end{definition}

    \begin{definition}[k-SAT]\label{ksat}
        A k-SAT problem is that, given a CNF formula $f$, in which each clause has exactly $K$ literals, decide whether there is an assignment satisfying all the clauses in $f$.
    \end{definition}

    \begin{theorem}\label{satthm}
        Suppose variables appear in a CNF formula $f$ are $\{x_1, \ldots, x_n\}$, and they can only be assigned by $0$ or $1$ with equal probability. If every variable appears in at most $\frac{2^k}{4k}$ clauses, there exists an assignment satisfying all the clauses in $f$.
    \end{theorem}
\end{tcolorbox}

\begin{proof}[proof of \cref{satthm}]
    Let $E_i$ be the event that the i-th clauses is wrong. Since every variable is assigned with equal probability, we have,
    \[
        P(E_i)\leq\frac{1}{2^k} =: p \mbox{ (in the Lovasz Local Lemma)}
    \]
    Consider the clauses as the vertices in a dependency graph, and two vertices have an edge only if they share a same variable. So in this dependency graph, we have, 
    \[
        \deg E_i \leq (\frac{2^k}{4k} - 1)k \leq \frac{2^k}{4} =: d \mbox{ (in the Lovasz Local Lemma)}
    \]
    Note that, \[4dp = 4 \frac{2^k}{4} \frac{1}{2^k} \leq 1
        \]
    So by \cref{LLL}, we have $P(\bigcap E_i^C) > 0$.
    
    Thus, there exists an assignment satisfying all the clauses in $f$.
\end{proof}


\bibliographystyle{plain}
\bibliography{ref}

\end{document}