\documentclass[a4paper, linespread=1.5]{article}
%\usepackage[UTF8]{ctex}
\usepackage{xeCJK}
\usepackage{geometry}
\usepackage{amsmath}
\usepackage{amssymb}
\usepackage{amsthm}
\usepackage{graphicx}
\usepackage{keyval}
\usepackage[dvipsnames,svgnames,x11names]{xcolor}
\usepackage{float}
\usepackage{ifthen}
\usepackage{calc}
\usepackage{ifplatform}
\usepackage{fancyvrb}
\usepackage{minted}
\usepackage{hyperref}
\usepackage{enumerate}
\usepackage{multicol}
\usepackage[all]{xy}
\usepackage{ulem}
\usepackage{epstopdf}
\usepackage{mathrsfs}
\usepackage{cancel}
\usepackage{algorithm}
\usepackage{algorithmic}
\usepackage{amsfonts}
\setlength{\parskip}{0.2\baselineskip}
\setlength{\parindent}{2em}
%\geometry{left=2.7cm,right=2.7cm,top=2.7cm,bottom=2.7cm}


\newtheorem{theorem}{Theorem}
\newtheorem{proposition}[theorem]{Proposition}
\newtheorem{lemma}[theorem]{Lemma}
\newtheorem{corollary}[theorem]{Corollary}
\newtheorem{definition}[theorem]{Definition}
\newtheorem{exercise}[theorem]{Exercise}

\newtheorem{innercustom}{\customname}
\providecommand{\customname}{}
\newcommand{\newcustomtheorem}[2]{
\newenvironment{#1}[1]
{
\renewcommand\customname{#2}
\renewcommand\theinnercustom{##1}
\innercustom
}
{\endinnercustom}
}
\newcustomtheorem{customthm}{Theorem}
\newcustomtheorem{customprop}{Proposition}
\newcustomtheorem{customlemma}{Lemma}
\newcustomtheorem{customcorollary}{Corollary}
\newcustomtheorem{customdef}{Definition}
\newcustomtheorem{customex}{Exercise}
\newcustomtheorem{customremark}{Remark}

\newcommand{\Natural}{\mathbb{N}}
\newcommand{\addbigcup}{\bigcup{\kern-1.12em{+}}\kern0.3em}
\newcommand{\nth}[1]{#1\textsuperscript{th}}

\begin{document}
    \title{ The existence of independent and identically distributed random variables }
    \author{金弘义\ 518030910333}
    \date{\today}
    \maketitle

    Before we prove the theorem, let's first introduce the concept of generalized inverse distribution function.
    \begin{definition}
        the generalized inverse distribution function $F^{-1}(p)=\inf\{x\in\mathbb{R}:F(x)\ge p\}$
    \end{definition}
    Here are some of its properties:
    \begin{itemize}
        \item [1)] $F^{-1}$ is non-decreasing
        \item [2)] $F^{-1}(F(x))\le x$
        \item [3)] $F(F^{-1}(p))\ge p$
        \item [4)] $F^{-1}(p)\le x$ if and only if $p\le F(x)$
        \item [5)] If $Y$ has a $U[0,1]$ distribution then $F^{-1}(Y)$ is distributed as $F$
    \end{itemize}
    \begin{proof}
        1.It comes obviously from the fact that $F$ is non-decreasing.

        2.$F^{-1}(F(x))=\inf\{y\in\mathbb{R}:F(y)\ge F(x)\} \le x$

        3.$F(F^{-1}(p))=F(\inf\{x\in\mathbb{R}:F(x)\ge p\}) \ge p$

        4. Impose $F$ on both sides of $F^{-1}(p)\le x$ and we get $F(F^{-1}(p))\le F(x)$.
            So $p\le F(F^{-1}(p))\le F(x)$ by property 3.
            It's similar for another side.

        5. We need to prove $P(F^{-1}(Y)\le x)=F(x)$.
        From property 4, we know $P(F^{-1}(Y)\le x)=P(Y\le F(x))$.
        Since Y is a uniform distribution, $P(Y\le F(x))=F(x)$.
        In conclusion, $P(F^{-1}(Y)\le x)=F(x)$.
    \end{proof}
    With the help of Skorokhod representation of random variables and the property 5 of generalized inverse function, we can simplify the problem to finding a countable sequence of independent random variables on $([0,1],\mathcal{B},Leb)$ which are uniformly distributed.

    \begin{customthm}{1}
        Given a certain distribution function, there exists a countable sequence of independent and identically distributed random variables on $([0,1],\mathcal{B},Leb)$.
    \end{customthm}
    \begin{proof}
        $\forall s\in [0,1]$, write it in 2-base (if there are multiple representation, choose the infinite one):
        \begin{align}
            s=\sum\limits_{i=1}^{+\infty}B_{i}(\frac{1}{2})^i
        \end{align}
        Claim $({B_i})_{i\ge 1}$ is independent.
        This has been proved in previous work "Independence of coin tossing events" by Haichen Dong.

        Let $p_n$ be the $n^{th}$ prime number and $I_n$ be $\{p_n^i:i\in\mathbb{N}\}$.
        It's obvious that $(I_n)$ don't intersect with each other.
        Denote $\phi_n(i)=p_n^i\in I_n$.
        Now we define $X_n=\sum\limits_{i=1}^{+\infty}B_{\phi_n(i)}\cdot(\frac{1}{2})^i$.
        $(X_n)$ is obviously independent.
        \begin{align*}
            \forall a\in [0,1] ,a&=\sum\limits_{i=1}^{+\infty}B_{i}'(\frac{1}{2})^i.\\
            P(X_n\le a)&=P(B_{\phi_n(1)}<B_{1}')+P(B_{\phi_n(1)}=B_{1}'\cap B_{\phi_n(2)}<B_{2}')+...\\
            &=\sum\limits_{i=1}^{+\infty}P(\bigcap\limits_{j=1}^{i-1}B_{\phi_n(j)}=B_{j}' \cap B_{\phi_n(i)}<B_{i}')\\
            &=\sum\limits_{i=1}^{+\infty}(\frac{1}{2})^{i-1}\cdot\frac{1}{2}B_{i}'\\
            &=a
        \end{align*}
        So $X_n$ has a uniform distribution on $[0,1]$.
        We can now derive that $F^{-1}(X_n)$ is an independent sequence on $([0,1],\mathcal{B},Leb)$ with distribution F.
    \end{proof}
\end{document}
