% This is a template for lecture notes.
\documentclass{article}
%\usepackage[UTF8]{ctex}
\usepackage{amssymb}
\usepackage{amsmath}
\usepackage{amsthm}
\usepackage{geometry}
\usepackage{booktabs}
\usepackage{bm}
\usepackage{tcolorbox}
%\CTEXoptions[today=old]
%Some commonly used notations
%\geometry{a4paper,bottom = 3cm,left = 3cm, right = 3cm}

%for reference
\usepackage{hyperref}
\usepackage[capitalise]{cleveref}
\crefname{enumi}{}{}

\newtheorem{theorem}{Theorem}
\newtheorem{lemma}[theorem]{Lemma}
\newtheorem{proposition}[theorem]{Proposition}
\newtheorem{corollary}[theorem]{Corollary}
\newtheorem{fact}[theorem]{Fact}
\newtheorem{definition}[theorem]{Definition}
\newtheorem{remark}[theorem]{Remark}
\newtheorem{question}[theorem]{Question}
\newtheorem{answer}[theorem]{Answer}
\newtheorem{exercise}[theorem]{Exercise}
\newtheorem{example}[theorem]{Example}
%\newenvironment{proof}{\noindent \textbf{Proof:}}{$\Box$}
\newtheorem{observation}[theorem]{Observation}

%to use newcommand for convenience
\newcommand\field{\mathbb{F}}
\newcommand\Real{\mathbb{R}}
\newcommand\Q{\mathbb{Q}}
\newcommand\Z{\mathbb{Z}}
\newcommand\complex{\mathbb{C}}

%this is how we define operators.
\DeclareMathOperator{\rank}{rank} % rank

\title{Doob's Optional Stopping Theorem and the ABRACADABRA Problem}
\author{Yi Gu}
\date{\today}

\begin{document}
    \maketitle
    This is a solution to Exercise 10.6 in the book,
    and the exercise is about a monkey typing random letters.

    \section{The problem}

    \begin{exercise}
      At each of times $1, 2, 3, \dots$, a monkey types a capital letter at random,
      the sequence of letters typed forming an IID sequence of random variables each chosen uniformly from amongst the $26$ possible capital letters.
      
      Just before each time $n = 1, 2, \dots$, a new gambler arrives,
      He bets \$$1$ that the $n$-th letter will be A.
      
      If he loses, he leaves.
      If he wins, he receives \$$26$ all of which he bets on the event that the $(n+1)$-th letter will be B.

      If he loses, he leaves.
      If he wins, he bets his whole fortune of \$$26^2$ that the $(n+2)$-th letter will be R.
      
      And so on through the ABRACADABRA sequence.

      Let $T$ be the first time by which the monkey has produced the consecutive sequence ABRACADABRA.
      Show that
      $$ E(T) = 26^11 + 26^4 + 26 $$
    \end{exercise}

    To solve this, a naive thought is that ABRACADABRA is $11$ lettes long, and the probability of a random $11$-letter word being ABRACADABRA is exactly $(\dfrac{1}{26})^{11}$,
    and if type $11$ letters is one trial,
    the expected number of trials is $26^{11}$.

    However, this idea broke up the random string into $11$-letter substrings,
    while this word may start in the middle of a block.
    In other words, we only consider strings whose starting position is divisible by $11$.

    \section{An intuitive idea}

    Suppose we open a casino, just next to the monkey and the typewriter.

    Before each keystroke, a gambler comes to our casino, and gamble in the same way as described in the problem statement.

    Suppose our casino is fair, i.e. the expected outcome is zero,
    which means if the gambler bets \$$1$, he will receive \$$26$ if he wins.

    We keep our casino until the monkey types ABRACADABRA for the first time,
    and denote the number of keystrokes up to this time as $T$.
    Our revenue so far will be exactly $T$ dollars.

    How much shall we pay for the gamblers?
    There is one gambler coming in just before the last keystroke, who wins \$$26$.
    Also there is one gambler winning his $4$ first bets ABRA, worthing \$$26^4$.
    Finally, the biggest winner wins \$$26^{11}$.
    No other gambler should be paid. 
    Thus our casino will have to pay the gamblers $26^{11} + 26^4 + 26$ dollars.

    Remember we are so kind that our casino is fair.
    Thus we have $$E(T) = 26^{11} + 26^4 + 26.$$

    \section{A formal proof}
    Let $(U_i)_{i \in \mathbb N}$ denote random letters drawn independently and uniformly from the English alphabet,
    i.e. $(U_i)_{i \in \mathbb N}$ are i.i.d. random variables, uniformly distributes in the set $\{\text{A}, \dots, \text{Z}\}$.

    Define $T$ as $T \mathrel{\mathop:}= \min\{n \in \mathbb N, n \geq 11 : U_{[n-10, n]} = \text{ABRACADABRA}\}$.

    Let $(\mathcal F_n)_{n \in \mathbb N_0}$ be the natural filtration of $(U_i)_{i \in \mathbb N}$.
    We are going to prove the following results.

    \begin{proposition}
      $T$ is a stopping times with $E(T) < \infty$.
    \end{proposition}

    \begin{proposition}
      There exists a martingale $M = (M_n)_{n \in \mathbb N_0}$ such that:
        \begin{enumerate}
          \item $M_0 = 0$, $M_T = 26^11 + 26^4 + 26 - T$;
          \item $\exists C \in (0, \infty)$ such that $|M_{n+1} - M_n| \leq C$, $\forall n \in \mathbb N_0$.
        \end{enumerate}
    \end{proposition}

    Combining these two proposition,
    one obtains immediately the result of the exercise
    by Doob's optional sampling theorem.

    \begin{proof}[Proof of proposition 2]
      $\{T \leq n\} = \bigcup\limits_{i=11}^n\{U_{[i-10, i]} = \text{ABRACADABRA}\} \in \mathcal F_n$.
      So $T$ is a stopping time.

      Let event $A_n$ defined as
      $$A_n \mathrel{\mathop:}= \{U_{[n+1, n+11]} = \text{ABRACADABRA}\}.$$
      As $U_i$ are independently and uniformly chosen letters, $P(A_n) = p^{11} > 0$ where $p = \dfrac{1}{26}$.

      Since $A_n \subseteq \{T \leq n + 11\}$,
      we have $P(T \leq n + 11 \mid T > n) \geq P(A_n \mid T > n)$.
      Note that $A_n$ and $\{T > n\}$ are independent,
      so $P(T \leq n+11 \mid T > n) \geq P(A_n) = p^{11}$,
      which is sufficient to have $E(T) < \infty$.
    \end{proof}

    \begin{proof}[Proof of proposition 3]
      Denote by $x_i$ the $i$-th letter in ABRACADABRA.
      The capital of the $j$-th gambler at time $n$ is given by the process $(K^j_n)_{n \in \mathbb N_0}$,
      where
      $$K^j_n = \begin{cases}
        1 & n < j\\
        K_{n-1} \cdot 26 \mathbf{1}_{U_n = x_n} & j \leq n \leq j + 10\\
        K_{11} & n > j + 10
      \end{cases}.$$
      And we define
      $$M_0 = 0, \quad M_n = \sum\limits_{j=1}^n(K^j_n - K^j_0) = (\sum\limits_{j=1}^n K^j_n) - n.$$

      We have $M_T = 26^{11} + 26^4 + 26 - T$,
      as $K_T^{T-10} = 26^{11}$, $K_T^{T-3} = 26^4$ and $K_T^T = 26$.

      For every fixed $j$, $(K_n^j)_{n \in \mathbb N_0}$ is a martingale as
      $$E[K_n^j \mid \mathcal F_{n-1}] = K_{n-1}^j \cdot 26 P(U_n = x_{n-j+1}) = K_{n-1}^j,$$
      because $U_n$ is independent of $\mathcal F_{n-1}$.

      As
      $$E[M_n \mid \mathcal F_{n-1}] = \sum\limits_{j=1}^n E[K_n^j \mid \mathcal F_{n-1}] - n = \sum\limits_{n-1}^j - n = \sum\limits{j=1}^{n-1}K_{n-1}^j + 1 - n = M_{n-1},$$
      $M$ is a martingale.

      Also not that $|K_n^j - K_{n-1}^j| \leq 25^{11}$,
      and $|M_n - M_{n-1}| \leq \sum\limits_{j=n-10}^n |K_n^j - K_{n-1}^j | \leq 11 \cdot 25^{11}$,
      i.e. $M$ has bounded increments.
    \end{proof}
\end{document}