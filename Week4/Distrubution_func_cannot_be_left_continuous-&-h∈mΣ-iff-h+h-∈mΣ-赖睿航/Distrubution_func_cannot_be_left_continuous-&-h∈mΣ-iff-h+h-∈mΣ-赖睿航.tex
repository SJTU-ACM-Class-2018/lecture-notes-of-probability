\documentclass[a4paper, linespread=1.5]{article}
%\usepackage[UTF8]{ctex}
\usepackage{xeCJK}
\usepackage{geometry}
\usepackage{amsmath}
\usepackage{amssymb}
\usepackage{amsthm}
\usepackage{graphicx}
\usepackage{keyval}
\usepackage[dvipsnames,svgnames,x11names]{xcolor}
\usepackage{float}
\usepackage{ifthen}
\usepackage{calc}
\usepackage{ifplatform}
\usepackage{fancyvrb}
\usepackage{minted}
\usepackage{hyperref}
\usepackage{enumerate}
\usepackage{multicol}
\usepackage[all]{xy}
\usepackage{ulem}
\usepackage{epstopdf}
\usepackage{mathrsfs}
\usepackage{cancel}
\usepackage{algorithm}
\usepackage{algorithmic}
\setlength{\parskip}{0.2\baselineskip}
\setlength{\parindent}{2em}
\geometry{left=2.7cm,right=2.7cm,top=2.7cm,bottom=2.7cm}


\newtheorem{theorem}{Theorem}
\newtheorem{proposition}[theorem]{Proposition}
\newtheorem{lemma}[theorem]{Lemma}
\newtheorem{corollary}[theorem]{Corollary}
\newtheorem{definition}[theorem]{Definition}
\newtheorem{exercise}[theorem]{Exercise}

\newtheorem{innercustom}{\customname}
\providecommand{\customname}{}
\newcommand{\newcustomtheorem}[2]{
    \newenvironment{#1}[1]
    {
        \renewcommand\customname{#2}
        \renewcommand\theinnercustom{##1}
        \innercustom
    }
    {\endinnercustom}
}
\newcustomtheorem{customthm}{Theorem}
\newcustomtheorem{customprop}{Proposition}
\newcustomtheorem{customlemma}{Lemma}
\newcustomtheorem{customcorollary}{Corollary}
\newcustomtheorem{customdef}{Definition}
\newcustomtheorem{customex}{Exercise}
\newcustomtheorem{customremark}{Remark}

\newcommand{\Natural}{\mathbb{N}}
\newcommand{\Real}{\mathbb{R}}
\newcommand{\addbigcup}{\bigcup{\kern-1.12em{+}}\kern0.3em}
\newcommand{\nth}[1]{#1\textsuperscript{th}}

\begin{document}
    \title{Distrubution Function Cannot be Left-Continuous\\\&\\$h\in m\Sigma$ iff $h^+, h^- \in m\Sigma$}
    \author{赖睿航\ 518030910422}
    \date{\today}
    \maketitle

    \begin{exercise}[Distrubution Function Cannot be Left-Continuous]
        Construct an example to show that the distribution function of a random variable may not be left-continuous.
    \end{exercise}

    \begin{proof}[Solution]
        Consider the probability space $(\Omega, \mathcal{F}, P) = ([0, 1], B[0, 1], Leb)$. And let $X(\omega)$ be a random variable which always takes $0$ for any real $\omega \in [0, 1]$. Now we try to calculate the distribution function $F_X(c)$ for $c \in \Real$. There are three cases:
        
        \begin{enumerate}
            \item If $c < 0$, then $F_X(c) = P(X \leqslant c) = P(\{\omega | X(\omega) \leqslant c\}) = P(\emptyset) = 0$ since there is no such $\omega$ with $X(\omega) < 0$.
            \item If $c = 0$, then $F_X(c) = F_X(0) = P(X \leqslant 0) = P(\{\omega | X(\omega) \leqslant 0\}) = P([0, 1]) = 1$ since for any $\omega \in [0, 1]$, $X(\omega) = 1$.
            \item If $c > 0$, then $F_X(c) = P(X \leqslant c) = P(\{\omega | X(\omega) \leqslant c\}) = P([0, 1]) = 1$ since for any $\omega \in [0, 1]$, $X(\omega) = 1$.
        \end{enumerate}
        
        Hence we have
        $$
        F_X(c) = \left\{
        \begin{aligned}
            0, c < 0, \\
            1, c \geqslant 0.
        \end{aligned}
        \right.
        $$
        It is clear that $F_X(c)$ is right-continuous but not left-continuous at $c = 0$, which finishes our proof.
    \end{proof}
    \vspace{5mm}
    
    \begin{exercise}[$h\in m\Sigma$ iff $h^+, h^- \in m\Sigma$]
        Let $(S, \Sigma)$ be a measurable space and take $h \in \Real^S$. Let $h^+ = \max(h, 0)$ and $h^- = \max(-h, 0)$. Show that $h \in m\Sigma$ if and only if $h^+, h^- \in m\Sigma$.
    \end{exercise}

    \begin{proof}
        We first prove from left to right. To show that $h^+, h^- \in m\Sigma$, we only need to show that $\{h^+ \leqslant c_1\} \in \Sigma$ for $\forall c_1 \in \Real$ and $\{h^- \geqslant c_2\} \in \Sigma$ for $\forall c_2 \in \Real$. There are two cases for each:
        \begin{enumerate}
            \item If $c_1 < 0$, $\{h^+ \leqslant c_1\} = \emptyset \in \Sigma$ since $h^+ \geqslant 0$ always holds.
            \item If $c_1 \geqslant 0$, $\{h^+ \leqslant c_1\} = \{h \leqslant c_1\} \in \Sigma$ since $h \leqslant c_1$ implies $h^+ = \max(h, 0) \leqslant c_1$.
            \item If $c_2 < 0$, $\{h^- \geqslant c_2\} = \emptyset \in \Sigma$ since $h^+ \geqslant 0$ always holds, too.
            \item If $c_2 \geqslant 0$, $\{h^- \geqslant c_2\} = \{h \geqslant -c_2\} \in \Sigma$ since $h \geqslant -c_2$ implies $h^- = \max(-h, 0) \leqslant c_2$.
        \end{enumerate}
        Hence we have $\{h^+ \leqslant c_1\} \in \Sigma$ for $\forall c_1 \in \Real$ and $\{h^- \geqslant c_2\} \in \Sigma$ for $\forall c_2 \in \Real$. So from $h \in m\Sigma$ we can know that $h^+, h^- \in m\Sigma$.
        
        Then we prove from right to left. Observe that:
        $$
        \begin{aligned}
            & h \geqslant 0 \Rightarrow h^+ = h, h^- = 0, \textrm{and} \\
            & h < 0 \Rightarrow h^+ = 0, h^- = -h.
        \end{aligned}
        $$
        So $h = h^+ - h^-$. Since $h^+, h^- \in m\Sigma$, we can conclude that $h \in m\Sigma$.
        
        Therefore, $h \in m\Sigma$ if and only if $h^+, h^- \in m\Sigma$.
    \end{proof}
\end{document}
