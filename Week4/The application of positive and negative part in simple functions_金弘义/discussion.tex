\documentclass[a4paper, linespread=1.5]{article}
%\usepackage[UTF8]{ctex}
\usepackage{xeCJK}
\usepackage{geometry}
\usepackage{amsmath}
\usepackage{amssymb}
\usepackage{amsthm}
\usepackage{graphicx}
\usepackage{keyval}
\usepackage[dvipsnames,svgnames,x11names]{xcolor}
\usepackage{float}
\usepackage{ifthen}
\usepackage{calc}
\usepackage{ifplatform}
\usepackage{fancyvrb}
\usepackage{minted}
\usepackage{hyperref}
\usepackage{enumerate}
\usepackage{multicol}
\usepackage[all]{xy}
\usepackage{ulem}
\usepackage{epstopdf}
\usepackage{mathrsfs}
\usepackage{cancel}
\usepackage{algorithm}
\usepackage{algorithmic}
\setlength{\parskip}{0.2\baselineskip}
\setlength{\parindent}{2em}
%\geometry{left=2.7cm,right=2.7cm,top=2.7cm,bottom=2.7cm}


\newtheorem{theorem}{Theorem}
\newtheorem{proposition}[theorem]{Proposition}
\newtheorem{lemma}[theorem]{Lemma}
\newtheorem{corollary}[theorem]{Corollary}
\newtheorem{definition}[theorem]{Definition}
\newtheorem{exercise}[theorem]{Exercise}

\newtheorem{innercustom}{\customname}
\providecommand{\customname}{}
\newcommand{\newcustomtheorem}[2]{
    \newenvironment{#1}[1]
    {
        \renewcommand\customname{#2}
        \renewcommand\theinnercustom{##1}
        \innercustom
    }
    {\endinnercustom}
}
\newcustomtheorem{customthm}{Theorem}
\newcustomtheorem{customprop}{Proposition}
\newcustomtheorem{customlemma}{Lemma}
\newcustomtheorem{customcorollary}{Corollary}
\newcustomtheorem{customdef}{Definition}
\newcustomtheorem{customex}{Exercise}
\newcustomtheorem{customremark}{Remark}

\newcommand{\Natural}{\mathbb{N}}
\newcommand{\addbigcup}{\bigcup{\kern-1.12em{+}}\kern0.3em}
\newcommand{\nth}[1]{#1\textsuperscript{th}}

\begin{document}
    \title{ The application of positive and negative part in constructing simple functions }
    \author{金弘义\ 518030910333}
    \date{\today}
    \maketitle
    
    \begin{customex}{9}
    	Let $(S,\Sigma)$ be a measurable space and take $h\in \mathbb{R}^{S} $.Let $h^{+}=max(h,0)$ and $h^{-}=max(-h,0)$. Show that $h\in m\Sigma$ if and only if $h^{+},h^{-}\in m\Sigma$.
        
    \end{customex}
	\begin{proof}[Solution]
		
		Observe that \begin{align*}
			h^{+}&=\begin{cases}
			0&h<0 \\
			h&h\ge 0
			\end{cases}\\
			h^{-}&=\begin{cases}
			-h& h<0\\
			0 & h\ge 0
			\end{cases}
		\end{align*}
		
		So we have
		 
		\begin{align*}
		h=h^{+}-h^{-}
		\end{align*}
		
		Since $m\Sigma$ is closed under taking sum and scalar multiplication, if $h^{+},h^{-}\in m\Sigma$, $h\in m\Sigma$. 
		
		Then we'll focus on another side. Assume $h\in m\Sigma$. Consider
		\begin{align*}
			\{h^{+}\le c\}=\begin{cases}
			\emptyset & c<0 \\
			\{h\le c\} & c\ge 0
			\end{cases}
		\end{align*}
		
		By the definition of $\sigma$-algebra, $\emptyset \in \Sigma$. $\{h\le c\}=h^{-1}(-\infty,c] \in \Sigma$. So $\{h^{+}\le c\}\in \Sigma$ $(\forall c\in \mathbb{R})$. We can derive that $h^{+}\in m\Sigma$.
		
		 $h^{-} \in m\Sigma$ can be derived similarly.
		
		In conclusion, $h\in m\Sigma$ if and only if $h^{+},h^{-}\in m\Sigma$.
	\end{proof}
	\begin{definition}
		Let $(S,\Sigma)$ be a measurable space. A function $f\in \mathbb{R}^S$ is a simple function with respect to $(S,\Sigma)$ provided it falls into the linear subspace of $\mathbb{R}^S$ spanned by  $\{\mathbf{1}_{A}:A\in \Sigma\}$. Note that every simple function is $\Sigma$-measurable. For each positive integer $n$, define the dyadic function $d_n\in \mathbb{R}^\mathbb{R}$ to be 
		\begin{align*}
			\sum\limits_{k=1}^{n2^n}\frac{k-1}{2^n}\mathbf{1}_{[\frac{k-1}{2^n},\frac{k}{2^n})}+n\mathbf{1}_{[n,\infty)}
		\end{align*}
	\end{definition}
	\begin{customex}{10}
		Take $f\in m\Sigma$. For each $n \in \mathbb{N}$, show that $f_{n}=d_{n}\circ f^{+}-d_{n}\circ f^{-}$ is a simple function with respect to $(S,\Sigma)$. Then illustrate that f is the limit of a sequence of simple functions.
	\end{customex}
    \begin{proof}[Solution]
    	Observe that
    	\begin{align*}
    		f_n(s)=\begin{cases}
    		\frac{k-1}{2^n}& f(s)\in [\frac{k-1}{2^n},\frac{k}{2^n}), k\in \mathbb{N}, 1\le k \le n2^n \\
    		n& f(s)\in [n,+\infty) \\
    		-n& f(s)\in (-\infty,-n] \\
    		-\frac{k-1}{2^n}& f(s)\in (-\frac{k}{2^n},-\frac{k-1}{2^n}], k\in \mathbb{N}, 1\le k \le n2^n
    		\end{cases}
    	\end{align*}
    	
    	which is equal to:
    	
    	\begin{align*}
   			f_n(s)=\begin{cases}
    		\frac{k-1}{2^n}& s\in f^{-1}[\frac{k-1}{2^n},\frac{k}{2^n}), k\in \mathbb{N}, 1\le k \le n2^n \\
    		n& s\in f^{-1}[n,+\infty) \\
    		-n&  s\in f^{-1}(-\infty,-n] \\
    		-\frac{k-1}{2^n}& s\in f^{-1} (-\frac{k}{2^n},-\frac{k-1}{2^n}], k\in \mathbb{N}, 1\le k \le n2^n
    		\end{cases}
    	\end{align*}
    	
    	Now we construct 
    	\begin{align*}
    		f_n=n\mathbf{1}_{f^{-1}[n,+\infty)}
    		-n\mathbf{1}_{f^{-1}(-\infty,-n]}
    		+\sum\limits_{k=1}^{n2^n}\frac{k-1}{2^n}\mathbf{1}_{f^{-1}[\frac{k-1}{2^n},\frac{k}{2^n})}
    		-\sum\limits_{k=1}^{n2^n}\frac{k-1}{2^n}\mathbf{1}_{f^{-1} (-\frac{k}{2^n},-\frac{k-1}{2^n}]}
    	\end{align*}
    	
    	Since 
    	\begin{align*}
    		f^{-1}[n,+\infty)&\in \Sigma \\
    		f^{-1}(-\infty,-n] &\in \Sigma \\
    		f^{-1}[\frac{k-1}{2^n},\frac{k}{2^n}) &\in \Sigma\\
    		f^{-1} (-\frac{k}{2^n},-\frac{k-1}{2^n}] &\in \Sigma
    	\end{align*}
    	
    	we can derive that $f_n$ is a simple function with respect to $(S,\Sigma)$.
    	
    	Without loss of generality, assume $f\ge0$.
    	Suppose $f<2^n$, then there exists $k\in\mathbb{N}$ and $1\le k \le n2^n$ such that $f \in[\frac{k-1}{2^n},\frac{k}{2^n}) $. We can then find that
    	\begin{align*}
    		f_n&=\frac{k-1}{2^n}\\
    		\forall \epsilon>0,\exists n_0=\lceil max(\log_{2}f,\log_{2}\frac{1}{\epsilon})\rceil+1&>0,\forall n>n_0,|f_n-f|<\frac{k}{2^k}-\frac{k-1}{2^n}=\frac{1}{2^n}<\epsilon
    	\end{align*}
    	
    	So $f$ is the limit of a sequence of simple functions $f_n$.
    	
    	
    \end{proof}
    
\end{document}
