\documentclass[a4paper, linespread=1.5]{article}
%\usepackage[UTF8]{ctex}
\usepackage{xeCJK}
\usepackage{geometry}
\usepackage{amsmath}
\usepackage{amssymb}
\usepackage{amsthm}
\usepackage{graphicx}
\usepackage{keyval}
\usepackage[dvipsnames,svgnames,x11names]{xcolor}
\usepackage{float}
\usepackage{ifthen}
\usepackage{calc}
\usepackage{ifplatform}
\usepackage{fancyvrb}
\usepackage{minted}
\usepackage{hyperref}
\usepackage{enumerate}
\usepackage{multicol}
\usepackage[all]{xy}
\usepackage{ulem}
\usepackage{epstopdf}
\usepackage{mathrsfs}
\usepackage{cancel}
\usepackage{algorithm}
\usepackage{algorithmic}
\setlength{\parskip}{0.2\baselineskip}
\setlength{\parindent}{2em}
%\geometry{left=2.7cm,right=2.7cm,top=2.7cm,bottom=2.7cm}


\newtheorem{theorem}{Theorem}
\newtheorem{proposition}[theorem]{Proposition}
\newtheorem{lemma}[theorem]{Lemma}
\newtheorem{corollary}[theorem]{Corollary}
\newtheorem{definition}[theorem]{Definition}
\newtheorem{exercise}[theorem]{Exercise}

\newtheorem{innercustom}{\customname}
\providecommand{\customname}{}
\newcommand{\newcustomtheorem}[2]{
    \newenvironment{#1}[1]
    {
        \renewcommand\customname{#2}
        \renewcommand\theinnercustom{##1}
        \innercustom
    }
    {\endinnercustom}
}
\newcustomtheorem{customthm}{Theorem}
\newcustomtheorem{customprop}{Proposition}
\newcustomtheorem{customlemma}{Lemma}
\newcustomtheorem{customcorollary}{Corollary}
\newcustomtheorem{customdef}{Definition}
\newcustomtheorem{customex}{Exercise}
\newcustomtheorem{customremark}{Remark}

\newcommand{\Natural}{\mathbb{N}}
\newcommand{\addbigcup}{\bigcup{\kern-1.12em{+}}\kern0.3em}
\newcommand{\nth}[1]{#1\textsuperscript{th}}

\begin{document}
    \title{ $\Sigma$-measurable ramp function}
    \author{金弘义\ 518030910333}
    \date{\today}
    \maketitle
    
    \begin{customex}{9}
    	Let $(S,\Sigma)$ be a measurable space and take $h\in \mathbb{R}^{S} $.Let $h^{+}=max(h,0)$ and $h^{-}=max(-h,0)$. Show that $h\in m\Sigma$ if and only if $h^{+},h^{-}\in m\Sigma$.
        
    \end{customex}
	\begin{proof}[Solution]
		
		Observe that \begin{align*}
			h^{+}&=\begin{cases}
			0&h<0 \\
			h&h\ge 0
			\end{cases}\\
			h^{-}&=\begin{cases}
			-h& h<0\\
			0 & h\ge 0
			\end{cases}
		\end{align*}
		
		So we have
		 
		\begin{align*}
		h=h^{+}-h^{-}
		\end{align*}
		
		Since $m\Sigma$ is closed under taking sum and scalar multiplication, if $h^{+},h^{-}\in m\Sigma$, $h\in m\Sigma$. 
		
		Then we'll focus on another side. Assume $h\in m\Sigma$. Consider
		\begin{align*}
			\{h^{+}\le c\}=\begin{cases}
			\emptyset & c<0 \\
			\{h\le c\} & c\ge 0
			\end{cases}
		\end{align*}
		
		By the definition of $\sigma$-algebra, $\emptyset \in \Sigma$. $\{h\le c\}=h^{-1}(-\infty,c] \in \Sigma$. So $\{h^{+}\le c\}\in \Sigma$ $(\forall c\in \mathbb{R})$. We can derive that $h^{+}\in m\Sigma$.
		
		 $h^{-} \in m\Sigma$ can be derived similarly.
		
		In conclusion, $h\in m\Sigma$ if and only if $h^{+},h^{-}\in m\Sigma$.
	\end{proof}
	
	
    
    
\end{document}
