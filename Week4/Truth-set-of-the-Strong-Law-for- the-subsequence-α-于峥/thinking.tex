\documentclass[UTF8]{ctexart}
\usepackage{amsmath}
\usepackage{amssymb}
\usepackage{amsthm}
\usepackage{graphicx}
\usepackage{CJK}
\usepackage{float}
\usepackage{mdframed}
\providecommand{\abs}[1]{\lvert#1\rvert}
\providecommand{\norm}[1]{\lVert#1\rVert}
\providecommand{\ud}[1]{\underline{#1}}

\newmdtheoremenv{thm}{Theorem}
\newmdtheoremenv{lemma}[thm]{Lemma}
\newmdtheoremenv{fact}[thm]{Fact}
\newmdtheoremenv{cor}[thm]{Corollary}
\newtheorem{eg}{Example}
\newtheorem{ex}{Exercise}
\newmdtheoremenv{defi}{Definition}
\newenvironment{sol}
  {\par\vspace{3mm}\noindent{\it Solution}.}
  {\qed \\ \medskip}

\newcommand{\ov}{\overline}
\newcommand{\ca}{{\cal A}}
\newcommand{\cb}{{\cal B}}
\newcommand{\cc}{{\cal C}}
\newcommand{\cd}{{\cal D}}
\newcommand{\ce}{{\cal E}}
\newcommand{\cf}{{\cal F}}
\newcommand{\ch}{{\cal H}}
\newcommand{\cl}{{\cal L}}
\newcommand{\cm}{{\cal M}}
\newcommand{\cp}{{\cal P}}
\newcommand{\cs}{{\cal S}}
\newcommand{\cz}{{\cal Z}}
\newcommand{\eps}{\varepsilon}
\newcommand{\ra}{\rightarrow}
\newcommand{\la}{\leftarrow}
\newcommand{\Ra}{\Rightarrow}
\newcommand{\dist}{\mbox{\rm dist}}
\newcommand{\bn}{{\mathbb N}}
\newcommand{\bz}{{\mathbb Z}}

\newcommand{\expe}{{\mathsf E}}
\newcommand{\pr}{{\mathsf{Pr}}}


\setlength{\parindent}{0pt}
%\setlength{\parskip}{2ex}
\newenvironment{proofof}[1]{\bigskip\noindent{\itshape #1. }}{\hfill$\Box$\medskip}

\theoremstyle{definition}
\newtheorem{problem}{Problem}
\newtheorem*{problem*}{Problem}

\pagenumbering{gobble}

\begin{document}


\title{Truth set of the Strong Law for the subsequence $\alpha$}
\author{于峥 518030910437}
\date{\today}
\maketitle

\begin{problem}
    Toss a fair coin infinitely often. The outcome of each toss
is either head (H) or tail (T) with equal probability. Take $\Omega = \{H,T\}^{\mathbb{N}}$.
which is the set of all possible outcomes of our infinitely many tosses.
A typical point $\omega \in \Omega$ is a sequence
$$
    \omega = (\omega_1, \omega_2, \dots), \omega_n \in \{ H,T\}.
$$
For any increasing sequence $(\alpha_n)$ of positive integers, let
$$
    F_{\alpha} = \{\omega : \frac {\#(k \leq n : \omega_{\alpha(k)}=H)} {n} \rightarrow \frac 1 2 \}
$$
Show that $\bigcap_\alpha F_\alpha = \emptyset$. \newline

\begin{sol}
    

We can think the $\alpha$ is a subsequence, and the $F_\alpha$ is the truth 
set of the Strong Law for the subsequence $\alpha$. 
If $\bigcap_\alpha F_\alpha \not= \emptyset$, it means that exists a $\omega$, its any subsequence satisfy
SLLN. But in fact for any $\omega$, we can construct a $\alpha$, make it unsatisfactory. 

For any 
$$
    \omega = (\omega_1, \omega_2, \dots), \omega_n \in \{ H,T\}.
$$

Define $A = \{ k : \omega_k = H \}$, $B = \{ k : \omega_k = T \}$. If A is 
infinite, we construct $\alpha$ from $A$. Because $A$ is well-ordered set, so we can let
 \begin{equation*}
    \alpha(n) = \left\{
    \begin{aligned}
        \min (A) && n = 1 \\
        \min(A \cap \bigcup_{k<n}\{\alpha(k)\} ) && n > 1
    \end{aligned}
    \right.
 \end{equation*}

Obviously, $alpha$ is a increasing sequence, and $\omega_{\alpha(n)} = H$ for 
all $n > 0$. So 

$$
    \frac {\#(k \leq n : \omega_{\alpha(k)}=H)} {n} \rightarrow 1
$$

and if $A$ is finite, we must have $B$ is infinite. Construct the function with the same way. 
We can get

$$
    \frac {\#(k \leq n : \omega_{\alpha(k)}=H)} {n} \rightarrow 0
$$
So $\forall \omega, \exists \alpha, s.t. \omega \not\in F_\alpha$. If $\bigcap_\alpha F_\alpha \not= \emptyset$, 
it means $\exists \omega, \forall \alpha \omega \in F_\alpha$. It is contradictory. Therefore

$$
\bigcap_\alpha F_\alpha = \emptyset
$$

\end{sol}
\end{problem}



\end{document}