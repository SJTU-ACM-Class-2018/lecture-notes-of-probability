% This is a template for lecture notes.
\documentclass{article}
% \usepackage[UTF8]{ctex}
\usepackage{amssymb}
\usepackage{amsmath}
\usepackage{amsthm}
\usepackage{geometry}
\usepackage{booktabs}
\usepackage{bm}
\usepackage{tcolorbox}
% \CTEXoptions[today=old]
%Some commonly used notations
%\geometry{a4paper,bottom = 3cm,left = 3cm, right = 3cm}

%for reference
\usepackage{hyperref}
\usepackage[capitalise]{cleveref}
\crefname{enumi}{}{}

\newtheorem{theorem}{Theorem}
\newtheorem{lemma}[theorem]{Lemma}
\newtheorem{proposition}[theorem]{Proposition}
\newtheorem{corollary}[theorem]{Corollary}
\newtheorem{fact}[theorem]{Fact}
\newtheorem{definition}[theorem]{Definition}
\newtheorem{remark}[theorem]{Remark}
\newtheorem{question}[theorem]{Question}
\newtheorem{answer}[theorem]{Answer}
\newtheorem{exercise}[theorem]{Exercise}
\newtheorem{example}[theorem]{Example}
%\newenvironment{proof}{\noindent \textbf{Proof:}}{$\Box$}
\newtheorem{observation}[theorem]{Observation}

%to use newcommand for convenience
\newcommand{\mbb}{\mathbb}
\newcommand{\mbf}{\mathbf}
\newcommand{\mbz}{\mathbb{Z}}
\newcommand{\mbn}{\mathbb{N}}
\newcommand{\mbp}{\mathbb{P}}
\newcommand{\mbh}{\mathbb{H}}
\newcommand{\mbq}{\mathbb{Q}}
\newcommand{\vep}{\varepsilon}
\newcommand{\rd}{\mathrm{d}}
\newcommand{\inv}{^{-1}}
\newcommand{\hp}{^\prime}
\newcommand{\mca}{\mathcal{A}}
\newcommand{\mcb}{\mathcal{B}}
\newcommand{\mcc}{\mathcal{C}}
\newcommand{\mcm}{\mathcal{M}}
\newcommand{\mcr}{\mathcal{R}}
\newcommand{\mcf}{\mathcal{F}}
\newcommand{\mfa}{\mathfrak{A}}
\newcommand{\mfb}{\mathfrak{B}}
\newcommand{\mfc}{\mathfrak{C}}
\newcommand{\mfi}{\mathfrak{I}}
\newcommand{\Iff}{\mbox{iff }}
\newcommand{\AND}{\mbox{ and }}

%this is how we define operators.
\DeclareMathOperator{\rank}{rank} % rank

\title{Notes on the Second Borel–Cantelli lemma}
\author{Lu Jiaxin\\
Student ID: 518030910412}
\date{\today}

\begin{document}
    \maketitle

\begin{tcolorbox}
    \begin{theorem}\label{thm1}
        Let $(y_n)_{n\in\mbn}$ be a sequence of reals from $[0,1]$ such that $\sum_{n\in\mbn} y_n = \infty$. Show that $\prod_{n\in\mbn}(1-y_n) = 0$.
    \end{theorem}  
\end{tcolorbox}

\begin{fact}\label{fact1}
    $\log(1-x) \leq -x$ for all $0\leq x \leq 1$.
\end{fact}

\begin{proof}
    From \cref{fact1} we have,
    \begin{align*}
        \log(\prod_{n\in\mbn}(1-y_n)) & = \sum_{n\in\mbn} \log(1-y_n) \\
        & \leq -\sum_{n\in\mbn} y_n
    \end{align*}
    Thus,
    \begin{align*}
        \prod_{n\in\mbn}(1-y_n) & = \exp(-\sum_{n\in\mbn} y_n)\\
        &= 0
    \end{align*}
\end{proof}


By using \cref{thm1}, we can prove the second Borel-Cantelli lemma (BC2).


\begin{tcolorbox}
    \begin{theorem}[Second Borel-Canteli lemma]\label{thm:BC2}
        Let $(E_n)$ be a sequence of events in a probability space $(\Omega, \mcf, P)$. If the events $E_n$ are pairwisely independent, then $\sum_{n\in\mbn} P(E_n) = \infty$ implies that $P(\lim \sup _{n \to \infty} E_n) = 1$.
    \end{theorem}  
\end{tcolorbox}

\begin{proof}
    Let $A = \lim \sup _{n \to \infty} E_n$. We shall prove that $P(A^C) = 0$. Let $B_i = \bigcap_{n=i}^{\infty} E_n^C$. Then $A^C = \bigcup_{i=1}^{\infty} B_i$. So, we shall prove that $P(B_i) = 0$ for all $i$. Now, for each $i$ and $k > i$,
    \begin{align*}
        P(B_i) &= P(\bigcap_{n=i}^{\infty} E_n^C) \\
        & \leq P(\bigcap_{n=i}^k E_n^C) = \prod_{n=i}^k[1-P(E_n)]
    \end{align*}
    Use \cref{thm1}, we can derive,
    \begin{align*}
        P(B_i) = \prod_{n=i}^k[1-P(E_n)] = 0
    \end{align*}
    Thus, $P(B_i) = 0$ for all $i \in \mbn$.        
\end{proof}

Comparing two Borel–Cantelli lemmas(BC1 and BC2), we find that for a sequence of pairwisely independent events $(E_n)$, we either have $P(\lim \sup _{n \to \infty} E_n) = 0$ or $P(\lim \sup _{n \to \infty} E_n) = 1$, depending on $\sum_{n\in\mbn} P(E_n)$. From materials I found, this is known as \textbf{Zero-one law}, which is widely used in probability theory.

\end{document}