% This is a template for lecture notes.
\documentclass[12pt]{article}
\usepackage{amssymb}
\usepackage[UTF8]{ctex}
\usepackage{amsmath}
\usepackage{amsthm}
\usepackage{geometry}
\usepackage{booktabs}
\usepackage{bm}
\usepackage{cite}
%\usepackage{CJK}
\usepackage[many]{tcolorbox}
%\tcbuselibrary{listingsutf8}
%\tcbuselibrary{skins, breakable, theorems, most}
%\geometry{a4paper,bottom = 3cm,left = 3cm, right = 3cm}
\CTEXoptions[today=old]
%for reference
\usepackage{hyperref}
\usepackage[capitalise]{cleveref}
\crefname{enumi}{}{}


\newtheoremstyle{mythm}{1.5ex plus 1ex minus .2ex}{1.5ex plus 1ex minus .2ex} 
    {}{\parindent}{\bfseries}{}{1em}{} 
\theoremstyle{mythm}
\newtheorem{theorem}{Theorem}
\newtheorem{lemma}[theorem]{Lemma}
\newtheorem{corollary}[theorem]{Corollary}
\newtheorem{fact}[theorem]{Fact}
\newtheorem{definition}[theorem]{Definition}
\newtheorem*{remark}{Remark}

%\newenvironment{proof}{\noindent \textbf{Proof:}}{$\Box$}

%to use newcommand for convenience
\newcommand\field{\mathbb{F}}
\newcommand\Real{\mathbb{R}}
\newcommand\Q{\mathbb{Q}}
\newcommand\Z{\mathbb{Z}}
\newcommand\complex{\mathbb{C}}
\newcommand\cc{\mathcal{C}}
\newcommand\uu{\mathcal{U}}
\newcommand\pp{\mathcal{P}}
\newcommand\ff{\mathcal{F}}
\renewcommand\refname{Reference}
\renewcommand{\proofname}{Proof}
\DeclareMathOperator{\range}{range}   

\title{Intersection of Uncountable a.s. Events May Be Empty}
\author{马浩博 518030910428}
\date{\today}
\begin{document}
\maketitle

\section*{Exercise 4}

  This exercise is going to say that intersection of uncountable events may be empty.
  We already know that for the countable situation, things are different:
  \begin{theorem}
  $$If \ F_n \in \mathcal{F} \ (n \in N) \ and \ P(F_n) = 1,\ \forall n, \ then $$
  $$P(\bigcap\limits_{n} F_n) = 1 $$
  \end{theorem}
  Now the number of $\alpha$ is obviously uncountable, so the theorem above is not working anymore.
  And we can get an opposite conclusion that
  $$\bigcap\limits_{\alpha} F_\alpha = \emptyset $$
\begin{proof}

  For any given $\omega$, we just need to find an $\alpha$ with $\omega \notin F_\alpha$.
  Then we can easily get the result that $\bigcap\limits_{\alpha} F_\alpha = \emptyset $.
  Thus the only thing to concern about is to find the $\alpha$.\\

  For $\omega = (\omega_1,\omega_2,\dots)$, we can assume without losing generality that 
  $$\forall m,\ \exists n > m \ \omega_n = H \ (n \in N)$$
  If $\omega$ doesn't satisify this proposition, then because $\omega = \{H,T\}$, we have $\forall m,\ \exists n > m \ \omega_n = T$.
  These two cases are just the same, so we only consider the first one.

  In this case, we can generate $\alpha$ in this way: For $m=0$, we can find $a_1>m$ with $\omega_{a_1} = H$.
  So we let $\alpha(1)=a_1$. Next, let $m=a_1$, find $a_2$ in the same way and also let $\alpha(2)=a_2$, and so on $\dots$ 
  Eventually, we get a map $\alpha$ with $\alpha(1)<\alpha(2)< \dots $ Obviously, 
  $$\frac{\#\{ k \le n: \ \omega_{\alpha(k)} = H \}}{n} \to 1$$
  Therefore we know that $\omega \notin F_\alpha$ and finish the proof.
\end{proof} 

\end{document}