\input{../preamble.tex}
\title{\textsc{Poincaré’s Recurrence Theorem for Incompressible
Transformations}}
\begin{document}
\maketitle
        
\begin{pro}
		Let $\left( \Omega ,\mathcal{F},P \right) $ be a probability space and let $T$ be
an incompressible transformation on it, namely it is measurable and
there is no $F \in \mathcal{F}$ such that $F\subset T^{-1}(F)$ and $P\left( F \right) < P\left( T^{-1}(F) \right) $.
Prove that $\forall E\in \mathcal{F}$ with $P\left( E \right) >0$, almost all points of $E$ return
to $E$ infinitely often under positive iterations by $T$.
\end{pro}
\begin{proof}
This problem is almost the same as the last one with some slight changes. 
Our goal is to prove 
\[
		P\left( E \setminus \limsup_{n \to \infty} T^{-n}(E) \right) = 0 
.\] 
The same as the proof of another version of PR theorem, the above is equivalent to prove
\[
		P\left( A_{k} \right) =0, \text{ in which } A_{k} = E \setminus \bigcup_{n\ge k} 
		T^{-n}(E)
.\] 
We will prove by making contradiction. Assume $P\left( A_{k} \right) > 0$, we have
\[
		P\left( B_{k} \right) =	P\left( E \cup \bigcup_{n\ge k} T^{-n}(E) \right) > 
		P\left( \bigcup_{n\ge k} T^{-n}(E)\right) = P\left( C_{k} \right) 
.\] 
Here $B_{k},C_{k}$ are notations for simplicity. Notice that the definition of incompressible 
transformation is exactly
\[
		F\subset T^{-1}\left( F \right)  \implies P(F) = P( T^{-1}(F))
.\]
Hence by $T^{-1}\left( C_{k-1} \right) = C_{k} $ we have 
\[
		P\left( C_{k} \right)  = P\left( C_{k-1} \right) 
.\] 
Similarly, as $T^{-1}\left( B_{k} \right) \subset C_{1}$ we derive that 
\[
		P\left( B_{k} \right)  = P\left( C_{1} \right) = P\left( C_{k} \right) 
.\]
Which leads to a contradiction.
\end{proof}
\end{document}
