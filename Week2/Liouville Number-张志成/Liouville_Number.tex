\documentclass[UTF8, 12pt]{ctexart}
\usepackage{enumitem}
\usepackage{amsmath}
\usepackage{amssymb}
\usepackage{amsfonts}
\usepackage{mathrsfs}
\usepackage{XCharter}
\usepackage{fancyhdr}
\setCJKmainfont{DENGL.TTF}
\usepackage{eulervm}
\usepackage{graphicx}
\usepackage{mdframed}
\usepackage{ntheorem}

\topmargin -.5in
\textheight 9in
\oddsidemargin -.25in
\evensidemargin -.25in
\textwidth 7in
\pagestyle{fancy}

\newenvironment{proof}{\noindent\ignorespaces\textbf{Proof:}}{\hfill $\square$\par\noindent}
\newenvironment{solution}{\noindent\ignorespaces\textbf{Solution:}}{\hfill $\square$\par\noindent}

\newtheorem{claim}{Claim}
\newtheorem{lemma}{Lemma}
\newtheorem*{theorem*}{Theorem}
\newtheorem*{exercise*}{Exercise}

\newenvironment{rcases}{\left.\begin{aligned}}{\end{aligned}\right\rbrace}

\title{Liouville Number}
\author{张志成 518030910439}
\date{\today}

\begin{document}
    \maketitle
    \begin{exercise*} 
        \begin{enumerate}[label=\arabic*)]
            \item Show that $\sum_{j\in\mathbb{N}}\frac{1}{2^{j!}}$ is a Liouville number.
            \item Show that every Liouville number must be transcendental.
        \end{enumerate}
    \end{exercise*}

    \begin{solution}
        \begin{enumerate}[label=\arabic*)]
            \item I present here a constructive proof. \\
                Since  $\sum_{j\in\mathbb{N}}\frac{1}{2^{j!}}$ is a sum of infinite series, we can choose, for every $n$, $\frac{p_n}{q_n}$ to be the prefix sum of $\frac{1}{2^{j!}}$. \par
                With this intuition, we let $$ \frac{p_n}{q_n} = \sum_{j = 1}^{n}\frac{1}{2^{j!}} $$
                Then, 
                \begin{align*}
                    |\sum_{j\in\mathbb{N}}\frac{1}{2^{j!}} - \frac{p_n}{q_n}| &= \sum_{j\in\mathbb{N}}\frac{1}{2^{j!}} - \sum_{j = 1}^{n}\frac{1}{2^{j!}} \\
                    &= \sum_{j = n + 1}^{\infty} \frac{1}{2^{j!}} \\
                    &< \sum_{j = (n + 1)!}^{\infty} \frac{1}{2^{j}} = \frac{1}{2^{(n+1)!-1}} \\
                    &< \frac{1}{(2^{n!})^n}\quad (n!\cdot n < n! \cdot (n+1) < (n+1)!)
                \end{align*}
                Thus, we choose $q_n$ to be $2^{n!}$, and $p_n$ to be $q_n * \sum_{j = 1}^{n}\frac{1}{2^{j!}} = 2^{n!} * \sum_{j = 1}^{n}\frac{1}{2^{j!}}$
            \item We prove by contradiction. \\
                Assume that there is a Liouville number $z$ that is not transcendental, then it must be a algebraic number of degree $n$.
                We know from the \textit{lecture notes} that, there exists a positive integer $M$ such that for all integers $p$ and $q$, 
                \begin{equation}
                |z - \frac{p}{q}| > \frac{1}{M\cdot q^n}
                \end{equation}
                It also holds for Liouville number $z$ that for every integer $n$ there exists integers $p$ and $q$ such that
                \begin{equation}
                |z - \frac{p}{q}| < \frac{1}{q^n}
                \end{equation}
                Notice that \textit{(1)} and \textit{(2)} are inequalities of opposite signs, in order to induce a contradiction, it must be the case that
                for some $n'$(not to be confused with the degree $n$),
                $$
                \frac{1}{q^{n'}} < \frac{1}{M\cdot q^n} \Longleftrightarrow M\cdot q^n < q^{n'}
                $$
                This could easily be achieved by setting
                $$
                n' = \lceil n + log_qM\rceil
                $$
                Thus a contradiction! \\ 
                Therefore, every Liouville number must be transcendental.
        \end{enumerate}
    \end{solution}
\end{document}