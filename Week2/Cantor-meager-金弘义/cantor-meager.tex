\documentclass[a4paper, linespread=1.5]{article}
%\usepackage[UTF8]{ctex}
\usepackage{xeCJK}
\usepackage{geometry}
\usepackage{amsmath}
\usepackage{amssymb}
\usepackage{amsthm}
\usepackage{graphicx}
\usepackage{keyval}
\usepackage[dvipsnames,svgnames,x11names]{xcolor}
\usepackage{float}
\usepackage{ifthen}
\usepackage{calc}
\usepackage{ifplatform}
\usepackage{fancyvrb}
\usepackage{minted}
\usepackage{hyperref}
\usepackage{enumerate}
\usepackage{multicol}
\usepackage[all]{xy}
\usepackage{ulem}
\usepackage{epstopdf}
\usepackage{mathrsfs}
\usepackage{cancel}
\usepackage{algorithm}
\usepackage{algorithmic}
\setlength{\parskip}{0.2\baselineskip}
\setlength{\parindent}{2em}
%\geometry{left=2.7cm,right=2.7cm,top=2.7cm,bottom=2.7cm}


\newtheorem{theorem}{Theorem}
\newtheorem{proposition}[theorem]{Proposition}
\newtheorem{lemma}[theorem]{Lemma}
\newtheorem{corollary}[theorem]{Corollary}
\newtheorem{definition}[theorem]{Definition}
\newtheorem{exercise}[theorem]{Exercise}

\newtheorem{innercustom}{\customname}
\providecommand{\customname}{}
\newcommand{\newcustomtheorem}[2]{
    \newenvironment{#1}[1]
    {
        \renewcommand\customname{#2}
        \renewcommand\theinnercustom{##1}
        \innercustom
    }
    {\endinnercustom}
}
\newcustomtheorem{customthm}{Theorem}
\newcustomtheorem{customprop}{Proposition}
\newcustomtheorem{customlemma}{Lemma}
\newcustomtheorem{customcorollary}{Corollary}
\newcustomtheorem{customdef}{Definition}
\newcustomtheorem{customex}{Exercise}
\newcustomtheorem{customremark}{Remark}

\newcommand{\Natural}{\mathbb{N}}
\newcommand{\addbigcup}{\bigcup{\kern-1.12em{+}}\kern0.3em}
\newcommand{\nth}[1]{#1\textsuperscript{th}}

\begin{document}
    \title{Properties of Cantor Set}
    \author{金弘义\ 518030910333}
    \date{\today}
    \maketitle
    
    \begin{customex}{1}
    	
        1) Show that the Cantor Set C is nowhere dense in $[0, 1]$.
        
        2) Find a meager set $T$ in $\mathbb{R}$ such that $T+T=\mathbb{R}$.
        
        3) Show that every subset of the real line $\mathbb{R}$ can be partitioned into two sets, one being of first category and the other being negligible. 
    \end{customex}
	\begin{proof}
		1) Recall the the explicit closed formulas for the Cantor set are:
		\begin{align*}
			C= [0,1]\setminus \bigcup\limits_{n=0}^\infty \bigcup\limits_{k=0}^{3^n-1}(\frac{3k+1}{3^{n+1}},\frac{3k+2}{3^{n+1}})
		\end{align*}
		
		It's obvious that $(\frac{3k+1}{3^{n+1}},\frac{3k+2}{3^{n+1}})$ is an open set. Since the union of open sets is an open set, the complement of $C$ is also open, which leads to the fact that $C$ is a closed set. So $\bar{C}=C$.
		
		Note that Cantor set can also be characterized as the set of all number in $[0,1]$ whose base-3 expansion doesn't contain any 1s. Assume there exists an interior point $x$ of $C$, then 
		there exists a ball $B$, with radius $r > 0$, which is contained in $C$. Since $C \subseteq [0,1]$, B is an interval, whose length is $2r$. Arbitarily choose $x_1 \in B$ and $x_2\in B$ such that $x_1<x_2$ and $x_2-x_1\ge r$. Let $n=\lceil\log_{\frac{1}{3}}r\rceil+2$.Consider $x_3=x_1+(\frac{1}{3})^n,x_4=x_1+2\cdot(\frac{1}{3})^n$. Since $(\frac{1}{3})^n<(\frac{1}{3})^{(\log_\frac{1}{3}r) +1}=\frac{1}{3}\cdot(\frac{1}{3})^{(\log_\frac{1}{3}r)}=\frac{1}{3}r$, $x_3,x_4\in B$.  It' s obvious that one of them conatins 1 in its base-3 expansion, which means one of them doesn't belong to Cantor set. This leads to a contradiction, or equivalently, that there is no interior point of $C$.
		
		Now we can safely conclude that $C$ is nowhere dense.
		
		2)Let 
		\begin{align*}
			T(n)&=\{a+n| a\in C\}\\ T&=\bigcup\limits_{n\in \mathbb{Z}}T(n)
		\end{align*}
		
		
		First We will prove $T$ is a meager set. Since $T(n)$ is constructed by translation of all points in $C$ by $n$, $T(n)$ is also nowhere dense. This leads to the fact that $T$ is a meager set, because $\mathbb{Z}$ is countable. 
		
		Then we prove $T+T=\mathbb{R}$. From $C+C=[0,2]$, we can similarly get $T(n)+T(n)=[2n,2n+2]$. Note that:
		\begin{align*}
			\bigcup\limits_{n\in\mathbb{Z}}(T(n)+T(n))\subseteq T+T \subseteq \mathbb{R}
		\end{align*} 
		However,
		\begin{align*}
			\bigcup\limits_{n\in\mathbb{Z}}(T(n)+T(n))=\bigcup\limits_{n\in\mathbb{Z}}[2n,2n+2]=\mathbb{R}
		\end{align*}
		Now we conclude that $T+T=\mathbb{R}$.
		
		3) Method 1:
		
		Pick an enumeration of $\mathbb{Q}$, named $q_n$. Let:
		\begin{align*}
			I_{i,j}&=(q_i-\frac{1}{2^{i+j+2}},q_i+\frac{1}{2^{i+j+2}})\\
			A_j&=\bigcup\limits_{i\in \mathbb{N}}I_{i,j}\\
			B&=\bigcap\limits_{j\in\mathbb{N}}A_j
		\end{align*}
		
		For any $\epsilon>0$, exist $j_0>0$ such that $|\frac{1}{2^{j_0}}|<\epsilon$. So we have :
		\begin{align*}
			|A_j|\le \sum_{i=0}^{\infty} \frac{1}{2^{i+j+1}}=\frac{1}{2^j} \\
			|B|\le|A_{j_0}|\le \frac{1}{2^{j_0}} < \epsilon
		\end{align*}
		
		This leads to that B is negligible. Then we prove $B^C$ is a meager set.
		\begin{align*}
			B^C=\bigcup\limits_{j\in\mathbb{N}}A_j^C
		\end{align*}
		
		Since $\mathbb{Q}\subseteq A_j$ and $\mathbb{Q}$ is dense, we can know that $A_j$ is dense. $A_j$ is also open because $A_j$ is the union of open intervals. These 2 properties of $A_j$ can lead to that $A_j^C$ is nowhere dense. So $B^C$ is a meager set, which finishes the proof.
		
		Method 2:
		Construct a sequence of fat cantor set by the following steps: To construct $C_n$($n\ge2$), first remove the middle $\frac{1}{2n}$ from the interval $[0,1]$. Then in the $i^{th}$ step(i starts from 2), remove subintervals of width $\frac{1}{2n\cdot4^{i-1}}$ from the middle of each of the $2^{i-1}$ remaining intervals. We can then calculate 
		\begin{align*}
			|C_n|&=1-\sum_{i=1}^{\infty}{\frac{1}{2n\cdot4^{i-1}}\cdot2^{i-1}=1-\frac{1}{n}}\\
			|C_n^C|&=\frac{1}{n}
		\end{align*}
		
		For any $\epsilon>0$, there exists $n_0>0$ such that $\frac{1}{n_0}<\epsilon$. Let $C= \bigcup\limits_{n=2}^{\infty}C_n$, then $C^C=\bigcap\limits_{n=2}^{\infty}C_n$. Note that $|C^C|\le|C_{n_0}^C|=\frac{1}{n_0}<\epsilon$, which leads to that $C^C$ is negligible in $[0,1]$.
		
		By construction,$C_n$ contains no intervals and therefore has empty interior, or equivalently, $C_n$ is nowhere dense in $[0,1]$. So $C$ is a meager set in $[0,1]$. 
		
		We can easily extend this conclusion to $\mathbb{R}$. Let $C'(i)=\{i+c | c\in C, i\in\mathbb{Z}\}$, $C'=\bigcup\limits_{i\in\mathbb{Z}}C'(i)$. It's easy to see that $C'$ is a meager set in $\mathbb{R}$ and $C'^C$ is negligible in $\mathbb{R}$.
		
		Method 3:
		Let $L$ be the set of Liouville numbers.
		Recall the first claim of Theorem 3 from notes on March 10, which is:
		Suppose that $\phi$ is positive. If $\sum_{q}\frac{1}{q\phi(q)}<\infty$, then $P(A_\phi)=0$. $A_\phi$ is the set of reals $x$ in $(0,1]$ such that $|x-\frac{p}{q}| < \frac{1}{q^2\phi(q)}$. Let $\phi(q)=q$, then $L \subseteq A_\phi$.
		Since $\sum_{q}\frac{1}{q^2}<\infty$, $P(A_\phi)=0$,which leads to that $P(L)=0$. Equivalently, $L$ is negligible. 
		
		Select an enumeration of $\mathbb{Q}$, named $q_n$.Let
		\begin{align*}
			U_n&=\bigcup\limits_{i=0}^{\infty}\{x :  q_i=\frac{p}{q},0<|x-\frac{p}{q}| < \frac{1}{q^n}\}\\
			L&=\bigcap\limits_{n=1}^{\infty}U_n
		\end{align*}
		
		$U_n$ is open, because it's the union of open sets. Since any open set that contains $\frac{p}{q}$ must intersect with $\{x :  0<|x-\frac{p}{q}| < \frac{1}{q^n}\}$ and $\mathbb{Q}$ is dense, it's easy to see that $U_n$ is dense according to the definition of "dense". So $U_n^C$ is nowhere dense, which means $L^C=\bigcup\limits_{n=1}^{\infty}U_n^C$ is a meager set. This finishes the proof.
	\end{proof}
	
	
    
    
\end{document}
