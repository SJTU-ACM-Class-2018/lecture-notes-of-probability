\input{D:/template.tex}
\author{Zhiyang Xun}
\title{Prove $\R$ Is Uncountable Without Using $\text{AC}_\omega$}

\begin{document}
 
\maketitle

\begin{tcolorbox}
\begin{theorem*} 
    $\R$ is an uncountable set and it doesn't rely on axiom of countable choice.
\end{theorem*} 
\end{tcolorbox}

\begin{proof}
Let $(a_i)$ be a sequence of reals.
Let $b_1 = a_1 + 1$, $c_1 = a_1 + 2$, so $a_1 \notin [b_1, c_1]$.
Then if $a_2 < \frac{1}{3}b + \frac{2}{3}c$, let $b_2 = b_1$ and $c_2 = \frac{2}{3}b_1 + \frac{1}{3}c_1$;
else we let $b_2 = \frac{1}{3}b_1 + \frac{2}{3}c_1$ and $c_2 = c_1$. 
In this way we guaranteed both $a_2 \notin [b_2, c_2]$ and $[b_2, c_2] \in [b_1, c_1]$.
Repeat this step and we get countable intervals $[b_i, c_i]$, satifying $a_i \notin [b_i, c_i]$ and $[b_{i+1}, c_{i+1}] \subseteq [b_i, c_i]$.
Finally, choose a point $a \in \cap_i[b_i, c_i]$ and we see that $a \notin \{a_i : i \in \N\}$.
\end{proof} 

This proof doesn't rely on the axiom of countable choice because we explictly gave a scheme to choose each $[a_i, b_i]$.

\end{document}