% This is a template for lecture notes.
\documentclass{article}
%\usepackage[UTF8]{ctex}
\usepackage{amssymb}
\usepackage{amsmath}
\usepackage{amsthm}
\usepackage{geometry}
\usepackage{booktabs}
\usepackage{bm}
\usepackage{tcolorbox}
\usepackage{indentfirst}
%\CTEXoptions[today=old]

%Some commonly used notations
%\geometry{a4paper,bottom = 3cm,left = 3cm, right = 3cm}

%for reference
\usepackage{hyperref}
\usepackage[capitalise]{cleveref}
\crefname{enumi}{}{}

\newtheorem{theorem}{Theorem}
\newtheorem{lemma}[theorem]{Lemma}
\newtheorem{proposition}[theorem]{Proposition}
\newtheorem{corollary}[theorem]{Corollary}
\newtheorem{fact}[theorem]{Fact}
\newtheorem{definition}[theorem]{Definition}
\newtheorem{remark}[theorem]{Remark}
\newtheorem{question}[theorem]{Question}
\newtheorem{answer}[theorem]{Answer}
\newtheorem{exercise}[theorem]{Exercise}
\newtheorem{example}[theorem]{Example}
%\newenvironment{proof}{\noindent \textbf{Proof:}}{$\Box$}
\newtheorem{observation}[theorem]{Observation}

%to use newcommand for convenience
\newcommand\field{\mathbb{F}}
\newcommand\Real{\mathbb{R}}
\newcommand\Q{\mathbb{Q}}
\newcommand\Z{\mathbb{Z}}
\newcommand\complex{\mathbb{C}}

%this is how we define operators.
\DeclareMathOperator{\rank}{rank} % rank

\title{0.012345678910... is simply normal to base 10}
\author{Ji Jiabao}
\date{\today}

\begin{document}
\maketitle
I can only prove 0.123456789..., known as the Champernowne Constant is simply
normal to base 10 now
\footnote{I also found the 
\href{https://londmathsoc.onlinelibrary.wiley.com/doi/abs/10.1112/jlms/s1-8.4.254}{original thesis} for Champernowne Constant, 
but i totally don't understand it after some reading}, below is the proof.

\begin{proof}
    $  $ \\
    \hspace*{1em} Let $N_n$ denotes the number of digits after we write the $a_n = 10^n (n \in \mathbb{Z})$th number 
    in $\mathbb{Z}$.\\
    \hspace*{1em} For $N_n$, we have a simple equation, by counting digits in different groups 
    divided by the number's length
    \begin{align*}
        N_n &= \sum_{i = 1}^{i = n} i (10^i - 10^{i - 1}) \\
            &= 9 \sum_{i = 1}^{i = n} i 10^{i - 1}\\
            &= (n - \frac{1}{9})10^n + \frac{1}{9}
    \end{align*}
    \hspace*{1em} For the number of $1$s for example, (2...9 is the same as 1) in the number we write, denoted as 
    $M_n$, we use induction.\\
    \hspace*{1em} $Base:$ \\
    \hspace*{2em} $n = 1$, $M_n = 1$\\
    \hspace*{1em} $Induction:$ After writing $N_{n+1}$ digits\\
    \hspace*{2em} In the first $N_n$ digits, we have $m_n$ 1s. As for the 1s between the $N_n + 1$ digit and $N_{n+1}$ digit,
    we write the integer number beween $10^n + 1$ and $10^{n + 1}$. Actually we just write the first $10^n$ integer for another 9 times
    and add $1, 2 ... 9$ to the highest digit.So we have.
    \begin{align*}
        M_{n + 1} & = 10M_n + 10^n
    \end{align*}
    \hspace*{1em} Based on the induction above, we can write the general formula of $M_n$ using high-school maths.
    \begin{align*}
        M_n = n10^{n - 1}
    \end{align*}
    
    Based on above, we get 
    \begin{align*}
        \lim_{n = N_n \rightarrow \infty } \frac{|x_1x_2...x_n|_d}{n} & = \frac{M_n}{N_n}\\
            &= \frac{n10^{n - 1}}{(n - \frac{1}{9})10^n + \frac{1}{9}}\\
            &= \frac{1}{10}
    \end{align*}
    \hspace*{1em} To finish the proof, we need to prove $\frac{|x_1x_2...x_n|_d}{n}$ decreases with $n$ increases,
    though not strictly, we can see the trend as $m$ is obviously much much smaller than $n$.\\
    \hspace*{1em} In all, 0.12345... is simply normal to base 10.

\end{proof}

\end{document}