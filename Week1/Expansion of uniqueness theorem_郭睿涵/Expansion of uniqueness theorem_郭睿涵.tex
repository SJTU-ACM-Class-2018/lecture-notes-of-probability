\documentclass[UTF8]{ctexart}
\usepackage{amsmath}
\usepackage{amssymb}
\usepackage{amsthm}
\usepackage{graphicx}
\usepackage{bm}
\usepackage{CJK}
\usepackage{float}
\usepackage{mdframed}

\usepackage{indentfirst}
\setlength{\parindent}{2em}

\providecommand{\abs}[1]{\lvert#1\rvert}
\providecommand{\norm}[1]{\lVert#1\rVert}
\providecommand{\ud}[1]{\underline{#1}}

\newmdtheoremenv{thm}{Theorem}
\newmdtheoremenv{lemma}[thm]{Lemma}
\newmdtheoremenv{fact}[thm]{Fact}
\newmdtheoremenv{cor}[thm]{Corollary}
\newtheorem{eg}{Example}
\newtheorem{ex}{Exercise}
\newmdtheoremenv{defi}{Definition}
\newenvironment{sol}
  {\par\vspace{3mm}\noindent{\it Solution}.}
  {\qed \\ \medskip}

\newcommand{\ov}{\overline}
\newcommand{\ca}{{\cal A}}
\newcommand{\cb}{{\cal B}}
\newcommand{\cc}{{\cal C}}
\newcommand{\cd}{{\cal D}}
\newcommand{\ce}{{\cal E}}
\newcommand{\cf}{{\cal F}}
\newcommand{\ch}{{\cal H}}
\newcommand{\cl}{{\cal L}}
\newcommand{\cm}{{\cal M}}
\newcommand{\cp}{{\cal P}}
\newcommand{\cs}{{\cal S}}
\newcommand{\cz}{{\cal Z}}
\newcommand{\eps}{\varepsilon}
\newcommand{\ra}{\rightarrow}
\newcommand{\la}{\leftarrow}
\newcommand{\Ra}{\Rightarrow}
\newcommand{\dist}{\mbox{\rm dist}}
\newcommand{\bn}{{\mathbb N}}
\newcommand{\bz}{{\mathbb Z}}

\newcommand{\expe}{{\mathsf E}}
\newcommand{\pr}{{\mathsf{Pr}}}


\setlength{\parindent}{0pt}
%\setlength{\parskip}{2ex}
\newenvironment{proofof}[1]{\bigskip\noindent{\itshape #1. }}{\hfill$\Box$\medskip}

\theoremstyle{definition}
\newtheorem{problem}{Problem}
\newtheorem*{problem*}{Problem}

\pagenumbering{gobble}

\begin{document}

\title{The Uniqueness Theorem of Measure Expansion from $(X,\bm{A})$ to $(X,\sigma(\bm{A}\cap\{C\}))$}
\date{Mar. 08, 2020}

\maketitle
\paragraph{Labels} In the following proof, we suppose
\begin{align*}
	\mu^{*}(C) = inf\{\mu(A): C\subset A, \bm{A}\}\text{  (outer measure)}\\
	\mu_{*}(C) = sup\{\mu(A): C\supset A, \bm{A}\}\text{  (inner measure)}
\end{align*}
\paragraph{Theorem} Suppose $\mu$ is a sigma-finite measure on $(X,\bm{A})$, $C\subset X$. If $\mu_1$ and $\mu_2$ are two expanding measure of $\mu$ from $(X,\bm{A})$ to $(X,\sigma(\bm{A}\cap\{C\}))$, which also satisfing
\begin{align*}
	\mu_i(C) = \mu^{*}(C) < +\infty,i=1,2
\end{align*}
Then $\mu_1=\mu_2$ on $\sigma(\bm{A}\cap\{C\})$
\paragraph{Proof} If $v$ is any expanding measure of $\mu$ from $(X,\bm{A})$ to $(X,\sigma(\bm{A}\cap\{C\}))$, and $v(C) = \mu^{*}(C) < +\infty$, there will be a $C_1 \in \bm{A}$, with both $C\subset C_1$, and 
\begin{align}
	\forall A\in \bm{A},v(A\cap C) = v(A\cap C_1)
\end{align}
Due to $\mu$ is finite on $\bm{A}$, there exists a measurable cover of $C$. Now we select any one of them, take it as $C_1$. In this way, $C\subset C_1,C_1\in\bm{A}$, and also $(C_1\setminus C)=0$. At the same time, refering to $v$ is an expanding measure of $\mu$, $v=\mu$ on $\bm{A}$. Then we get to know
\begin{align*}
	v(C_1) &= \mu(C_1) = \mu_{*}(C \cup (C_1 \setminus C))\notag\\
	&\leq \mu^{*}(C) + \mu_{*}(C_1 \setminus C) = \mu^{*}(C)\notag\\
	&\leq \mu(C_1) = v(C_1) 
\end{align*}
So we've got $\mu^{*}(C) = v(C_1)$ and so that $v(C) = v(C_1)$\\
Considering $v(C) = \mu^{*}(C) < +\infty$, we now know that
\begin{align*}
	v(A\cap C) &\leq v(A\cap C_1) = v(A\cap C) + v(A\cap(C_1\subset C))\\         &\leq v(A\cap C) + v(C_1\subset C) = v(A\cap C)
\end{align*}
So formula (1) is correct.\\
\subparagraph{Due} to $\mu_1$ and $\mu_2$ are two expanding measure of $\mu$ from $(X,\bm{A})$ to $(X,\sigma(\bm{A}\cap\{C\}))$ which satisfying $\mu_i(C) = \mu^{*}(C) < +\infty,i=1,2$, we can get the following formula :
\begin{align}
	\forall A\in \bm{A},\mu_1(A\cap C) = \mu_1(A\cap C_1)\\
	\forall A\in \bm{A},\mu_2(A\cap C) = \mu_2(A\cap C_1)
\end{align}
Because of $\mu_1 = \mu_2 = \mu$ on $\bm{A}$, $A\cap C_1 \in \bm{A}$
\begin{align}
	\mu_1(A\cap C_1) = \mu_2(A\cap C_1)
\end{align}
According formula (2),(3) and (4), we've got
\begin{align}
	\mu_1(A\cap C) = \mu_2(A\cap C)
\end{align}
Besides that, we also know
\begin{align*}
	\mu_1(A\cap C) \leq \mu_1(C) = \mu^{*}(C) < +\infty,i=1,2
\end{align*}
So that,
\begin{align}
	\mu_1(A\cap C^{c})=\mu_2(A\cap C^{c})
\end{align}
So there exists $A_1,A_2 \in \bm{A}$, such that
\begin{align*}
	A = (A_1\cap C)\cup(A_2\cap C^{c})
\end{align*}
By considering formula (5) and (6), we know that
\begin{align*}
	\mu_1(A) &= \mu_1[(A_1\cap C)\cup(A_2\cap C^{c})]\\
             &= \mu_1(A_1\cap C)+\mu_2(A_2\cap C^{c})\\
             &= \mu_2(A_1\cap C)+\mu_2(A_2\cap C^{c})\\
             &= \mu_2[(A_1\cap C)\cup(A_2\cap C^{c})]\\
             &= \mu_2(A)
\end{align*}
Which means now we have proved that $\mu_1 = \mu_2$ on $\sigma(\bm{A}\cap\{C\})$
\end{document}

