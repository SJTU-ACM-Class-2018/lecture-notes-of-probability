% This is a template for lecture notes.
\documentclass{article}
\usepackage[UTF8]{ctex}
\usepackage{amssymb}
\usepackage{amsmath}
\usepackage{amsthm}
\usepackage{geometry}
\usepackage{booktabs}
\usepackage{bm}
\usepackage{tcolorbox}
\CTEXoptions[today=old]
%Some commonly used notations
%\geometry{a4paper,bottom = 3cm,left = 3cm, right = 3cm}

%for reference
\usepackage{hyperref}
\usepackage[capitalise]{cleveref}
\crefname{enumi}{}{}

\newtheorem{theorem}{Theorem}
\newtheorem{lemma}[theorem]{Lemma}
\newtheorem{axiom}[theorem]{Axiom}
\newtheorem{proposition}[theorem]{Proposition}
\newtheorem{corollary}[theorem]{Corollary}
\newtheorem{fact}[theorem]{Fact}
\newtheorem{definition}[theorem]{Definition}
\newtheorem{remark}[theorem]{Remark}
\newtheorem{question}[theorem]{Question}
\newtheorem{answer}[theorem]{Answer}
\newtheorem{exercise}[theorem]{Exercise}
\newtheorem{example}[theorem]{Example}
%\newenvironment{proof}{\noindent \textbf{Proof:}}{$\Box$}
\newtheorem{observation}[theorem]{Observation}

%to use newcommand for convenience
\newcommand\field{\mathbb{F}}
\newcommand\Real{\mathbb{R}}
\newcommand\Q{\mathbb{Q}}
\newcommand\Z{\mathbb{Z}}
\newcommand\complex{\mathbb{C}}

%this is how we define operators.
\DeclareMathOperator{\rank}{rank} % rank

\title{An Introduction for Axiom of Choice}
\author{Yao Yuan}
\date{\today}

\begin{document}
\maketitle

\begin{tcolorbox}
    Bertrand Russell stated that "to choose one sock from each of infinitely many pairs of socks requires the Axiom of Choice, but for shoes the Axiom is not needed." 
    
    What does Russell mean?
\end{tcolorbox}

Remark: this article is somewhat like an introduction of Axiom of Choice instead of serious proofs. In this article, I would like to talk not only about the explanation of Russell's statement, but also some understandings on Axiom of Choice.

Now suppose you have some pairs of socks. You have to choose one of each pair of socks. It is pretty obvious you can really do that. Now we increase the number of socks to infinite, can we still make such a chioce?

If you don't find the difference between the finite case and the infinite case, just think about this quesion: "how can you make this choice?" Think of they're not apples, but more abstract things, like all the intervals on $\Real$ with length greater than $0$, can you still really choose one element from each of them? 

To make the above seemingly correct things acceptable, mathematicians made the famous Axiom -- the Axiom of Choice. 

\begin{axiom}\label{axiom:AC}
    Every indexed collection of non-empty sets has a choice function.
\end{axiom}

Axiom of choice has an equivalent statement, as follows.

\begin{proposition}
    Every surjective map has a right inverse.
\end{proposition}

Note here: every injective map has a left inverse does NOT rely on Axiom of Choice.

A very common misunderstanding in Axiom of Choice among those who know little about math, is that they consider the Axiom of Choice is for choosing an element from an infinite set. Be carefull. Axiom of choice is only needed when you choose elements each from infinite sets.

Here's another case, not that trivial, that does not rely on axiom of choice. Axioms of choice are not necessary when it is possible to specify an explicit choice. That is to say, if there exist rules in the set family to choose a specific element for each set, then the Axiom of Choice is not applied.

That's the true essence of Russell's example of shoes. For each pair of shoes, we can define a rule only to choose the left shoe but exclude the right one. Then we do not need to apply Axiom of Choice.

In the end of this article, I would like to add an example of the usage of AC to make this article complete. When we try to prove $\Real$ is uncountable, we prove by contradiction and apply the Cantor's diagonal process. Actually, in that process, when we construct the number not in our list, we choose a number for infinite digits, and that relies on Axiom of Choice.
\end{document}