% This is a template for lecture notes.
\documentclass{article}
\usepackage[UTF8]{ctex}
\usepackage{amssymb}
\usepackage{amsmath}
\usepackage{amsthm}
\usepackage{geometry}
\usepackage{booktabs}
\usepackage{bm}
\usepackage{tcolorbox}
\CTEXoptions[today=old]
%Some commonly used notations
%\geometry{a4paper,bottom = 3cm,left = 3cm, right = 3cm}

%for reference
\usepackage{hyperref}
\usepackage[capitalise]{cleveref}
\crefname{enumi}{}{}

\newtheorem{theorem}{Theorem}
\newtheorem{lemma}[theorem]{Lemma}
\newtheorem{proposition}[theorem]{Proposition}
\newtheorem{corollary}[theorem]{Corollary}
\newtheorem{fact}[theorem]{Fact}
\newtheorem{definition}[theorem]{Definition}
\newtheorem{remark}[theorem]{Remark}
\newtheorem{question}[theorem]{Question}
\newtheorem{answer}[theorem]{Answer}
\newtheorem{exercise}[theorem]{Exercise}
\newtheorem{example}[theorem]{Example}
%\newenvironment{proof}{\noindent \textbf{Proof:}}{$\Box$}
\newtheorem{observation}[theorem]{Observation}


\newtheorem{problem}{Problem}
\newtheorem*{problem*}{Problem}
\newtheorem{solution}{Solution}
\newtheorem*{solution*}{Solution}

%to use newcommand for convenience
\newcommand\field{\mathbb{F}}
\newcommand\Real{\mathbb{R}}
\newcommand\Q{\mathbb{Q}}
\newcommand\Z{\mathbb{Z}}
\newcommand\complex{\mathbb{C}}

%this is how we define operators.
\DeclareMathOperator{\rank}{rank} % rank

\title{Discussions on some quizzes}
\author{柳志轩 518030910426}
\date{\today}

\begin{document}
    \maketitle
\begin{problem}
	中原大战,冯玉祥问士兵:空中飞机多还是乌鸦多?众人答:乌鸦多。冯再问:然则
	乌鸦拉屎掉到你们头上没有?众人异口同声:没有。冯说:所以嘛,随着飞机投下的
	炸弹的命中机会就更少了,大家莫怕!
\end{problem}

\begin{solution}
	考虑被炸弹炸到的概率和被乌鸦拉屎砸到的概率,是一件很有意思的事情。如果不考虑乌鸦聚集地(如树林)被砸的概率较高和飞机扔炸弹一般往目标密集的地方扔,炸弹(屎)的覆盖范围不会重叠等因素,假设乌鸦和飞机均在该固定区域随机拉屎或扔炸弹,则p(飞机)=num(炸弹数)*S(炸弹覆盖面积)/S(区域面积),p(乌鸦)=num(拉屎数)*S(拉屎的覆盖面积)/S(区域面积),考虑到飞机的数量与乌鸦的数量,炸弹覆盖面积与屎的覆盖面积等,实际上被炸弹砸到的概率确实低于被乌鸦拉屎砸到的概率。但是飞机扔炸弹不可能随机扔,而鸟类一般与人类都会保持距离,所以冯玉祥的话是缺乏根据的。
\end{solution}

\begin{problem}
	太阳连续升起了 100 天。明天照常升起的概率有多大?
\end{problem}

\begin{solution}
	拉普拉斯提出该类问题的正确答案是$\frac{n+1}{n+2}$,推导过程使用了拉普拉斯概率计算的第六第七原则。有兴趣的同学可以看看$https://blog.csdn.net/three_body/article/details/14498889$该博客。
\end{solution}

\begin{problem}
	对于生男生女,比如生男生女概率各 50%,每个家庭都生到第一个男孩就不再生,那
	么产生的男女比例是多少?
\end{problem}

\begin{solution}
	这的确是一个反直觉的问题。按照直觉我们会认为男孩的概率会大一些,但实际上男女孩的概率是一样的。此题想要想明白可以这样考虑:在父母们生第一个孩子时,男女孩的概率相同,在第一个孩子中男女数量相同。生了女孩的父母继续生第二个孩子的时候,在第二个孩子中男女数量也相同,不断重复,可以发现最终总的男女孩数量还是相同的。也即男女比例仍是1:1的关系。
\end{solution}



\end{document}