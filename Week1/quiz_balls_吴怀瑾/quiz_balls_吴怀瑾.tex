% This is a template for lecture notes.
\documentclass{article}
\usepackage[UTF8]{ctex}
\usepackage{amssymb}
\usepackage{amsmath}
\usepackage{amsthm}
\usepackage{geometry}
\usepackage{booktabs}
\usepackage{bm}
\usepackage{tcolorbox}
\CTEXoptions[today=old]
%Some commonly used notations
%\geometry{a4paper,bottom = 3cm,left = 3cm, right = 3cm}

%for reference
\usepackage{hyperref}
\usepackage[capitalise]{cleveref}
\crefname{enumi}{}{}

\newtheorem{theorem}{Theorem}
\newtheorem{lemma}[theorem]{Lemma}
\newtheorem{proposition}[theorem]{Proposition}
\newtheorem{corollary}[theorem]{Corollary}
\newtheorem{fact}[theorem]{Fact}
\newtheorem{definition}[theorem]{Definition}
\newtheorem{remark}[theorem]{Remark}
\newtheorem{question}[theorem]{Question}
\newtheorem{answer}[theorem]{Answer}
\newtheorem{exercise}[theorem]{Exercise}
\newtheorem{example}[theorem]{Example}
%\newenvironment{proof}{\noindent \textbf{Proof:}}{$\Box$}
\newtheorem{observation}[theorem]{Observation}

%to use newcommand for convenience
\newcommand\field{\mathbb{F}}
\newcommand\Real{\mathbb{R}}
\newcommand\Q{\mathbb{Q}}
\newcommand\Z{\mathbb{Z}}
\newcommand\complex{\mathbb{C}}

%this is how we define operators.
\DeclareMathOperator{\rank}{rank} % rank

\title{Quiz Balls}
\author{吴怀瑾 518030910414}
\date{\today}

\begin{document}
    \maketitle
\section{}
\begin{question}{}{}
    罐子里有 70 个黑球和 30 个白球。每次从中取一个球直到罐子中只含单色球为止。最后罐子中剩的都是白球的概率为多少?
\end{question}

\paragraph{solution}
My intuition of the answer is $\frac {3}{10}$. We can discuss this question in a more general way.
Suppose there are $X$ black balls and $Y$ white balls  in the pot initially. First, We have $P = \binom {X+Y} Y = \frac {(X+Y)!}{X!Y!}$ different ways to 
take all the balls out of the pot. Next we consider how many different ways when last $k$ balls are exactly white which means the last $(k+1)^{th}$ ball must be
black, suppose the number is $w_i$, and the sum is $W$.
\begin{align*}
    W = \sum_{i=1}^{Y} w_i &= \sum_{i=1}^{Y} \binom {X+Y-1-i} {X-1}\\
                           &= \sum_{i=1}^{Y} ( \binom {X+Y-i} {X} - \binom {X+Y-i-1} {X}) \\
                           &= \binom {X+Y-1} {X}\\
                           &= \frac{(X+Y-1)!}{X!(Y-1)!}
.\end{align*}
Every way to take out the whole balls has equal probability. So the answer of the question is $\frac W P = \frac {Y} {X+Y}$.
\end{document}