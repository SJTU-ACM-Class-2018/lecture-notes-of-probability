% This is a template for lecture notes.
\documentclass{article}
\usepackage[UTF8]{ctex}
\usepackage{amssymb}
\usepackage{amsmath}
\usepackage{amsthm}
\usepackage{geometry}
\usepackage{booktabs}
\usepackage{bm}
\usepackage{tcolorbox}
\CTEXoptions[today=old]
%Some commonly used notations
%\geometry{a4paper,bottom = 3cm,left = 3cm, right = 3cm}

%for reference
\usepackage{hyperref}
\usepackage[capitalise]{cleveref}
\crefname{enumi}{}{}

\newtheorem{theorem}{Theorem}
\newtheorem{lemma}[theorem]{Lemma}
\newtheorem{proposition}[theorem]{Proposition}
\newtheorem{corollary}[theorem]{Corollary}
\newtheorem{fact}[theorem]{Fact}
\newtheorem{definition}[theorem]{Definition}
\newtheorem{remark}[theorem]{Remark}
\newtheorem{question}[theorem]{Question}
\newtheorem{answer}[theorem]{Answer}
\newtheorem{exercise}[theorem]{Exercise}
\newtheorem{example}[theorem]{Example}
%\newenvironment{proof}{\noindent \textbf{Proof:}}{$\Box$}
\newtheorem{observation}[theorem]{Observation}

%to use newcommand for convenience
\newcommand\field{\mathbb{F}}
\newcommand\Real{\mathbb{R}}
\newcommand\Q{\mathbb{Q}}
\newcommand\Z{\mathbb{Z}}
\newcommand\complex{\mathbb{C}}
\newenvironment{myproof}{\ignorespaces\paragraph{Proof:}}{\hfill $\square$\par\noindent}
%this is how we define operators.
\DeclareMathOperator{\rank}{rank} % rank

\title{dual Cantor-Bernstein theorem}
\author{李孜睿 518030910424}
\date{\today}


\begin{document}
	\maketitle
	(dual Cantor-Bernstein theorem). Let $f \in Y^X$ and $g \in X^Y$ be two surjectve maps. Assuming AC, show that there is a bijection $h \in Y^X$ such that $h\subseteq f\cup g^{-1}$.
	\begin{myproof}
		Given any surjections $f\in Y^X$ and $g\in X^Y$, by AC (we can get a Right inverse of a surjective map) there are injections $u\subseteq f^{-1}$ and $v\subseteq g^{-1}$.
		
		By Cantor-Bernstein Theorem, there exits a bijection $h\in Y^X$ such that $h\subseteq u^{-1}\cup v$.
		
		As $u^{-1}\subseteq f$ and $v\subseteq g^{-1}$, $h\subseteq f\cup g^{-1}$.
	\end{myproof}
\end{document}