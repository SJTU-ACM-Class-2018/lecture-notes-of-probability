% This is a template for lecture notes.
\documentclass{article}
\usepackage[UTF8]{ctex}
\usepackage{amssymb}
\usepackage{amsmath}
\usepackage{amsthm}
\usepackage{geometry}
\usepackage{booktabs}
\usepackage{bm}
\usepackage{tcolorbox}
\CTEXoptions[today=old]
%Some commonly used notations
%\geometry{a4paper,bottom = 3cm,left = 3cm, right = 3cm}

%for reference
\usepackage{hyperref}
\usepackage[capitalise]{cleveref}
\crefname{enumi}{}{}

\newtheorem{theorem}{Theorem}
\newtheorem{lemma}[theorem]{Lemma}
\newtheorem{proposition}[theorem]{Proposition}
\newtheorem{corollary}[theorem]{Corollary}
\newtheorem{fact}[theorem]{Fact}
\newtheorem{definition}[theorem]{Definition}
\newtheorem{remark}[theorem]{Remark}
\newtheorem{question}[theorem]{Question}
\newtheorem{answer}[theorem]{Answer}
\newtheorem{exercise}[theorem]{Exercise}
\newtheorem{example}[theorem]{Example}
%\newenvironment{proof}{\noindent \textbf{Proof:}}{$\Box$}
\newtheorem{observation}[theorem]{Observation}

%to use newcommand for convenience
\newcommand\field{\mathbb{F}}
\newcommand\Real{\mathbb{R}}
\newcommand\Q{\mathbb{Q}}
\newcommand\Z{\mathbb{Z}}
\newcommand\complex{\mathbb{C}}
\newenvironment{myproof}{\ignorespaces\paragraph{Proof:}}{\hfill $\square$\par\noindent}
%this is how we define operators.
\DeclareMathOperator{\rank}{rank} % rank

\title{dual Cantor-Bernstein theorem}
\author{李孜睿 518030910424}
\date{\today}


\begin{document}
	\maketitle
	1. (Cantor-Bernstein theorem) Let $f \in Y^X$ and $g \in X^Y$ be two injective maps. Then there is a bijection $h \in Y^X$ such that $h\subseteq f\cup g^{-1}$.
	\begin{myproof}
		Let $C_0 = X \backslash g(Y)$, $C_{n+1} = g(f(C_n))$. And $$C = \bigcup_{n=0}^{\infty}C_n$$
		
		For every $x\in X$, define
		$$h(x) = \left\{
			\begin{array}{lr}
				f(x),&x\in C\\
				g^{-1}(x),&x\notin C
			\end{array}\right.$$
		
		We can easily see $h\subseteq f\cup g^{-1}$ from the definition. Next, prove $h$ is bijective.
		
		$h$ is injective: Assume $a\neq b\wedge h(a)=h(b)$(h is not injective). If $a\in C \wedge b\in C$, $h(a) = f(a) \neq f(b) = h(b)$. If $a\notin C \wedge b\notin C$, $h(a) = g^{-1}(a) \neq g^{-1}(b) = h(b)$. If $a\in C \wedge b\notin C$, $g^{-1}(b) = f(a)\Rightarrow b = g(f(a))\Rightarrow b\in C$, contradicting to $b\notin C$. Otherwise is the same. All cases contradict to the premise. So $h$ is injective.
		
		$h$ is surjective: For any $y\in Y$, If $g(y)\in C$, there is a certion $n\geq 1$ such that $g(y)\in C_n$. Also, there is a $x\in C_{n-1}\subseteq X$ that $h(x) = f(x) = y$. If $g(y)\notin C$, there is $g(y)\in X$ that $h(g(y))=g^{-1}(g(y))=y$. Thus $h$ is surjective.
		
		Above all, $h\in Y^X$ is a bijection and $h\subseteq f\cup g^{-1}$.
	\end{myproof}
	
	2. (dual Cantor-Bernstein theorem) Let $f \in Y^X$ and $g \in X^Y$ be two surjectve maps. Assuming AC, show that there is a bijection $h \in Y^X$ such that $h\subseteq f\cup g^{-1}$.
	\begin{myproof}
		Given any surjections $f\in Y^X$ and $g\in X^Y$, by AC (we can get a Right inverse of a surjective map) there are injections $u\subseteq f^{-1}$ and $v\subseteq g^{-1}$.
		
		By Cantor-Bernstein Theorem, there exits a bijection $h\in Y^X$ such that $h\subseteq u^{-1}\cup v$.
		
		As $u^{-1}\subseteq f$ and $v\subseteq g^{-1}$, $h\subseteq f\cup g^{-1}$.
	\end{myproof}
\end{document}