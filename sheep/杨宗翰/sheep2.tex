\documentclass[11pt,a4paper,oneside]{article}

\usepackage[utf8]{inputenc}
\usepackage[english]{babel}
\usepackage{olymp,clrscode3e}
\usepackage{amsmath,graphicx,epigraph,bnf,enumerate,comment,listings,fontspec,color,indentfirst}
\usepackage{subcaption}
\usepackage{listings}
\usepackage{ctex}
\usepackage{CJK}

\newcommand{\chr}[1]{\mbox{`\texttt{#1}'}}
\newcommand{\txt}[1]{\mbox{``\texttt{#1}''}}

\renewcommand{\contestname}{
Magical Sheep (2), MS107 Probability Homework, Yang Zonghan, \today
}

\lstset{
    breaklines=true,  %代码过长则换行
    numbers=left, %行号在左侧显示
	basicstyle = \fontspec{Consolas},
    commentstyle=\color{gray}, %注释颜色
    frame=shadowbox%用方框框住代码块
}
\setCJKmainfont[BoldFont={黑体}, ItalicFont={华文楷体}, BoldItalicFont={黑体}]{华文中宋}
\begin{document}
\title{\textbf{Magical Sheep(2)}}
\author{Yang Zonghan(518030910435)}
\date{\today}
\setlength{\parindent}{2em}

吴润哲哥哥说得非常对,我来补充几句。

\section{$O(n)$ solving Markov Chain}
首先,$O(n ^ 3)$ 的高斯消元是可以优化的。本质上,消元解决的是 Markov 链问题,可以抽象为 $$
\begin{cases}
    x_1 = L \\
    x_n = R \\
    x_i = a_i x_{i- 1} + b_i x_{i + 1}, &i \in (1, n)
\end{cases}
$$ 而消元方程组的系数是形如下的,只有 $O(n)$ 个:
$$\begin{bmatrix}
     1 & 0 & 0 & 0 &\cdots & 0 & L\\
     a_2 & -1 & b_2 & 0 & \cdots & 0 & 0 \\
     0 & a_3 & -1 & b_3 &\cdots 0& & 0 \\
     \vdots & \vdots & \vdots  & \vdots  & \vdots & \ddots & \vdots \\
     0 & 0 & \cdots & a_{n - 1} & -1 & b_{n - 1} & 0\\
     0 & 0 & \cdots & 0 & 0 & 1 & R
\end{bmatrix}$$ 因此,这个矩阵消元复杂度可以减少为 $O(n ^ 2)$

但是其实我们可以做到 $O(n)$。对于方程 $$
x_i = a_i x_{i- 1} + b_i x_{i + 1}
$$ 可以变形为 $$
x_{i + 1} = \frac {x_i - a_i x_{i - 1}} {b_{i}}
$$

因此对于这样的 Markov 链我们可以用 $O(n)$ 的时间递推,用 $x_1, x_2$ 将所有项表达出来。算出 $x_n$ 的表达我们就能带入计算得到 $x_2$,进而得到所有值了。

\section{About monotonicity}

如果我们承认了决策意义上的单调性,也即,承认了 $i + j = \text{const}$ 时,在 $i$ 由小到大时 $f_{i, j}$ 的决策存在一个分割点,前半段决策为取 $\max$,后半段决策为取 Markov 链更新,那么我们可以一层更新做到 $O(n)$,即用上述更新方法倒着维护值,直到取 $\max$ 更优为止。总复杂度 $O(n ^ 2)$。(当然其实不承认这种单调性也是能 $O(n ^ 2)$ 做的,随时维护随时取 $\max$ 即可,正确性应该不依赖于这个单调性)

但是这个单调性,直觉上很对,并且实践证明很对,但囿于我的水平不行,没法给出一个完整的解释。

尽管这个 $f_{i, j}$ 的本身的单调性非常优秀,
\begin{gather*}
    f_{i, j} \leq f_{i + 1, j} \\
    f_{i, j} \leq f_{i, j + 1} 
\end{gather*}
\begin{center}
	\includegraphics[width=100mm]{sheepsheet.jpg}
\end{center}

并且我们可以在 $\frac i {i + j}$ 充分小时,可以知道某个前缀必然是取 $\max$:对于上图的表格中的任意一列,即 $i = \text{const}$ 时,当 $p, q > 0$ 时,取 $i + j \rightarrow \infty$,有
\begin{gather*}
    \frac {ip} {ip + jq} \rightarrow 0 \\
    \frac {jq} {ip + jq} \rightarrow 1
\end{gather*}
如果 $f_{i, j}$ 的决策为取 $\max$ 而 $f_{i + 1, j - 1}$ 取 Markov 链更新,那么
$$ f_{i + 1, j - 1} \rightarrow f_{i, j} = f_{i, j - 1} \leq f_{i + 1, j - 2}$$

这是一个矛盾,因此每一列充分高的位置都是取 $\max$ 的。同理可证充分高的每一列充分右的位置是取 Markov 链更新的。这两种情况对应白羊太多和黑羊太多的情况。

但我,我还是不会证明决策点对于每一行每一列唯一。

而且,根据吴润哲哥哥的结果,这样的分割点\emph{大概}就是 $ip - jq \leq 0$——和直觉分析相符的结果。但其中存在几个畸变点。这几个畸变点我想了很久,也没有办法给出解释。

也许只好寄希望于吴老师或其他助教、同学来解答这些疑惑了。或者我想,学习了本学期的内容或许我就能说出其中的道理了吧。
\end{document}