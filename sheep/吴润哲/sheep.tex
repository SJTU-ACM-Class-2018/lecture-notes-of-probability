% !TeX spellcheck = en_US
\documentclass{article}

\usepackage[UTF8]{ctex}
\usepackage{geometry}
\usepackage{amsthm}
\usepackage{amsmath}
\usepackage{amssymb}
\usepackage{enumerate}
\usepackage{graphicx}  

\geometry{a4paper, left = 2cm, right = 2cm, top = 2cm}

\newcommand\hE{\mathbb{E}}
\newcommand\hF{\mathbb{F}}
\newcommand\bC{\mathbf{C}}
\newcommand\bN{\mathbf{N}}
\newcommand\bQ{\mathbf{Q}}
\newcommand\bR{\mathbf{R}}
\newcommand\bZ{\mathbf{Z}}
\newcommand\hZ{\mathbb{Z}}
\newcommand\lproof{\item "$\Leftarrow$":}
\newcommand\rproof{\item "$\Rightarrow$":}

\title{关于``施了魔法的羊''的一些想法}
\date{}

\begin{document}
	
	\maketitle
	
	因为没有多少相关知识,才疏学浅,所以我只好用我比较熟悉的方式来切入这个问题试试看......
	
	\section{模型与算法}
	
	因为管理者的决策只和当前时刻有多少只黑羊,有多少只白羊有关,所以我尝试着用概率结合上动态规划来研究这个问题。既然决策只和黑白羊各自的个数有关,我用$(a,b)$表示一种有$a$只黑羊,$b$只白羊的状态。记$f(a,b)$表示有$a$只黑羊和$b$只白羊的情况,在使用最优的策略下,最终黑羊的期望个数。
	
	如果管理者选择在最开始的时候移走若干只白羊(至少一只),设管理者一次操作之后还剩下$c$只白羊$(0\le c< b)$,那么就存在转移
	$$f(a,b)\leftarrow \frac{ap}{ap+cq}f(a+1,c-1)+\frac{cq}{ap+cq}f(a-1,c+1)$$
	
	管理者想要找到最优的策略,就要找到所有转移里面最大的那一个,也就是
	$$f(a,b)\leftarrow \max\{a, \max_{0< c<b}\{\frac{ap}{ap+cq}f(a+1,c-1)+\frac{cq}{ap+cq}f(a-1,c+1)\}\eqno{(1)}$$
	
	这个转移是无后效性的,也就是只要按照$a+b$从小到大计算所有的$f(a,b)$,那么在算$f(a,b)$的时候,$f(a+1,c-1)$和$f(a-1,c+1)$都已经算好了。所以这个转移可以用递推很快算出来。之所以用了``$\leftarrow$''而不是``='',是因为这个状态转移是不全面的,因为管理者还有可能一只羊也不移走,也就是接下来的这种情况。
	
	如果不移走任何羊,那么在有$a$只黑羊,$b$只白羊的情况下。经过足够长的时间,总会有羊发生变化。即在经过足够久之后,有$\frac{ap}{ap+bq}$的概率让一只白羊变黑,有$\frac{bq}{ap+bq}$的概率让一只黑羊变白。对于这种不移走任何羊的情况,可以列出以下转移。
	
	$$f(a,b)\leftarrow\frac{ap}{ap+bq}f(a+1,b-1)+\frac{bq}{ap+bq}f(a-1,b+1)\quad(a,b>0)\eqno{(2)}$$
	
	这个方程用到了$f(a,b),f(a+1,b-1),f(a-1,b+1)$,它们的两个自变量之和总是常数a+b,也就是说转移出现了环,所以不能像(1)式那样直接递推。所以我只好退而求其次找一个复杂度高一点的算法。仔细想一下,管理者如果不移走任何羊,那么状态可能会从$(a,b)$,变成$(a+1,b-1)$,然后又变到$(a+2,b-2)$,接着可能又变到$(a+1,b-1)$等等。这样不断变化,最终一定会遇到以下两种情况之一
	\begin{enumerate}
		\item 达到了某个状态$(a',b')$,满足$(a+b=a'+b',a',b'>0)$,这时候最优策略要求管理者要移走白羊了。也就是说不移走羊的情况结束了。
		\item 达到了$(0,a+b)$或者$(a+b,0)$。整个过程完全结束了。
	\end{enumerate}
	
	所以我们可以直接枚举两个数$l,r(l<a<r)$,表示$(l,a+b-l)$和$(r,a+b-r)$分别是离$(a,b)$最近的两个会让不移走羊的情况结束的状态。这时候只需要算一下从$(a,b)$开始,不移走任何羊,最终状态变成$(l,a+b-l)$的概率$p_1$和$(r,a+b-r)$的概率$p_2$(当然$p_1+p_2=1$)。这个用方程组就可以算了,后面再提。因此这部分转移是这样的
	
	$$f(a,b)\leftarrow p_1\cdot g(l,a+b-l)+p_2\cdot g(a+b-r,r)\eqno{(3)}$$
	
	其中$g(a,b)$表示有$a$只黑羊和$b$只白羊的情况,刚开始必须先移走至少一只白羊,在最优的策略下,最终黑羊的期望个数。
	
	这时候我们再回来考虑管理者必须要移走白羊的那一部分转移(也就是(1)式做的事情),发现利用$g(a,b)$,$(1)$式可以简化成如下形式
	$$f(a,b)\leftarrow g(a,b)\eqno{(4)}$$
	
	现在只需要知道$g(a,b)$怎么算就行了。注意到$g(a,b)$直接使用(1)式就可以递推出来,写下来也就是
	$$g(a,b)= \max\{a, \max_{0< c<b}\{\frac{ap}{ap+cq}g(a+1,c-1)+\frac{cq}{ap+cq}g(a-1,c+1)\}\eqno{(5)}$$
	
	万事俱备,接下来只需要计算出$p_1,p_2$就好了。记$p(a,b)$表示从$(a,b)$开始不移走任何白羊,变成$(l,a+b-l)$的概率(换句话说就是从$(a,b)$开始的$p_1$)。注意到
	$$p(a,b)=\frac{ap}{ap+bq}p(a+1,b-1)+\frac{bq}{ap+bq}p(a-1,b+1)$$
	
	发现$p(a,b)$之间的转移有环,也不能递推。但是没有关系,我们可以列出如下$r-l+1$个方程
	
	$$\begin{cases}
		p(l,a+b-l)=1\\
		\cdots\\
		p(a-1,b+1)=\frac{(a-1)p}{(a-1)p+(b+1)q}\cdot p(a,b)+\frac{(b+1)q}{(a-1)p+(b+1)q}\cdot p(a-2,b+2)\\
		p(a,b)=\frac{ap}{ap+bq}\cdot p(a+1,b-1)+\frac{bq}{ap+bq}\cdot p(a-1,b+1)\\
		p(a+1,b-1)=\frac{(a+1)p}{(a+1)p+(b-1)q}\cdot p(a+2,b-2)+\frac{(b-1)q}{(a+1)p+(b-1)q}\cdot p(a,b)\\
		\cdots\\
		p(r,a+b-r)=0
	\end{cases}$$
	
	恰好有$r-l+1$个未知数$p$,所以用高斯消元等办法解上述方程就可以把需要的$p_1,p_2$算出来了。
	
	至此,运用(3)(4)(5)式以及上面这个方程组,就搭建了一个模型,并且有了一个时间复杂度为$O(n^6)$的算法。后来发现$r$只可能取$a+b$,以及方程组也很有特点,所以复杂度可以降到$O(n^4)$甚至$O(n^3)$。对于2000只黑羊和2000只白羊的情况,应该可以运行出结果来。不过如果只是用来观察和验证猜想,就应该没有必要优化速度了。(代码附后)	
	\section{验证}
	
	以上是粗略的理论分析,感觉上还没有十足的把握(毕竟我有时候想问题也比较粗心),所以我按照书上的结论验证了一下。书上有说,当不考虑$p,q$,即$p,q=1$的时候$$\lim_{k\rightarrow \infty} f(k,k)-(2k+\frac{\pi}{4}-\sqrt{\pi k})=0$$
	
	虽然我看不太懂它的证明,但是拿来用还是很好的。如果$k=50$,此时$2k+\frac{\pi}{4}-\sqrt{\pi k}\approx 88.25$。用上面的算法算了一下,结果是88.283351,比较准确。我还写了一个在$p,q=1$的情况下按照书上说的``保证黑羊总比白羊多''的策略的模拟器,进行了20000000次模拟,结果和上面的算法运行出来的结果很吻合。说明这个模型和算法应该没什么问题,至少在$p,q=1$的时候应该是对的。
	
	\section{探究与结论}
	
	因为对于$p,q=1$,或者说$p=q$的情况,书上已经给出了策略,就是时刻保证黑羊总比白羊多(当然因为知识有限我没有看懂证明...),所以很自然地猜在$p,q$不一定相同的情况下,只需要让产生黑羊的概率比产生白羊的大就好了,也就是保证$$\frac{ap}{ap+bq}-\frac{bq}{ap+bq}>0\eqno{(*)}$$
	
	我用上面提到的算法和模拟器进行了测试。模拟器的策略是:一旦$\frac{ap}{ap+bq}-\frac{bq}{ap+bq}\le 0$,就开始不断移走白羊。
	
	粗试了几组,结果都比较吻合,但是我渐渐发现了有点问题。例如,在有21只黑羊,15只白羊,p=0.5,q=0.8的时候,动态规划算法给出的结果为:最优策略下黑羊个数的期望值是28.971949,但是我的模拟器在多次模拟的情况下,结果稳定在28.964附近。如果我们默认算法得出的结果是最佳值的话,这就意味着模拟器采用的这种策略可能不是最优的。
	
	事实上,我把算法里面发生的转移打印出来就发现了问题。用表格整理如下(规定$p=0.5,q=0.8$)
	
	\begin{figure}[htb]
		\centering
		\includegraphics[width=18cm]{1.PNG}
	\end{figure} 
	
	这个表格的第$i$行第$j$列表示:有$i$只黑羊,$j$只白羊的时候,$ip-jq$的值(也就是$(*)$式左边的值)。
	
	如果第$i$行第$j$列是红色字体,就表示在有$i$只黑羊,$j$只白羊的情况下,根据我猜测出来的结论,因为$ip-jq<0$了,所以管理者的最优策略是需要移走白羊的。如果字体是黑色则相反,不需要移走白羊。
	
	如果第$i$行第$j$列是黄色背景的黑字,就表示根据上面提到的算法,管理者如果要采取最优策略,就必须要移走至少一只白羊。也就是说,黄色背景的位置,就是我的猜测和算法算出来的结果之间有出入的地方。如果认为算法是对的,并且不认为精度误差会出问题,那么我最开始的猜测就是错的。
	
	进一步地,我观察了一下这个表中黄色的位置,有了另一个猜测,即实际上管理者的最佳策略应该是这样的:管理者选取了一个函数$h(a,b)$,满足$h(a,b)$略小于$ap-bq$。一旦黑羊个数a和白羊个数b使得$h(a,b)<0$了,就不断移走白羊。
	
	但是$h(a,b)$具体是什么我就不知道了......不过,从结果上看,凭借$ap-bq<0$来作为一个策略应该已经算比较好了吧,尽管不是最优的$\sim$以我现有的能力只能解决到此,希望在以后的学习过程中能有新的发现。
	
	运行动态规划算法的代码:https://paste.ubuntu.com/p/jtQmHrj8Cb/
	
	运行模拟器的代码:https://paste.ubuntu.com/p/BNx9Qpfcv8/

\end{document}