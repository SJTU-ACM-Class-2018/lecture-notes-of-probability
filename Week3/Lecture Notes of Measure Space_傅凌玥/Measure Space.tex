\input{/Users/fulingyue/Desktop/def.tex}
\author{Fu Lingyue}
\title{Notes of Measure Space}
\date{\today}
\begin{document}
\maketitle

\section{Measure Theory and Topology}

Space is often a set with a certain structure or operation, rather than a simple set.
\paragraph{Topology} Topology deals with open sets. 

We have a set T. T has some open subsets $O$.  If open sets satisfy some requirements in $U = \{O\}$, then $(T,U)$ is a topology space. T is the origin space. U is a subset of $2^T$(set consisting of subsets of T).

U should satiafy:
\begin{enumerate}
  \item $T$ and $\varnothing$ is open set.
  \item $\forall O_i \in U,  \bigcap_{i=1}^N O_i$ is open set (must be finite).
  \item $\forall O_i \in U, \bigcup O_i$ is open set (arbitrary number).
  
\end{enumerate}

We define \textbf{continuous function} $f$ on T as 
$$\forall G \text{ is an open set}, f^{-1}(G) \text{ is an open set as well}.$$

Measure Space has many similar definitions.
\paragraph{Measurable Space} Similarly, measurable space is a dualistic group $(X,\mathcal F)$. $X$ is our origin discussion space, and $\mathcal F$ is a subset of $2^X$, which satisfies some conditions as follows: 
\begin{enumerate}
  \item $\varnothing, X \in \mathcal F$.
  \item If $E\in \mathcal F$, then $E^C\in \mathcal F$.
  \item If $E_i\in \mathcal F$, then $E_1 \cup E_2$ as well.
\end{enumerate}

Those elements in $\mathcal F$ is measurable. The measurable space $(X,\mathcal F)$ also can be described as "$\mathcal F$ is a $\sigma-$Algebra on $X$".

\paragraph{Measure Space} Measure spaces define a measure $m$ on measurable space. Then measure space is a triple $(X,\mathcal F, m)$. 

(if $\mathcal F$ is too big, we cannot find a measure function $m$ intuitively)

The measure $m: \mathcal F\mapsto [0,\infty]$ has to satisfy:
\begin{enumerate}
\item $m(\varnothing) = 0.$
\item If $F_i\cap F_j = \varnothing(i\not= j)$ and $F = \cup F_n \in \mathcal F$, then 
$$m(F) = \Sigma m(F_n).$$  
\end{enumerate}

\begin{corollary}
  (Additive) If $F,G \in \mathcal F,F \cap G = \varnothing$,  $m(F\cup G) = m(F) + m(G)$.
\end{corollary}




\section{$\sigma$-Algebra}
$\sigma$-Algebra has three basic rules, and some properties are derived from them.
\begin{corollary}
  These two corollary is useful!
  \begin{enumerate}
  \item $\bigcup_{i=1}^n E_i\in \mathcal F(n\in \mathbb N)$ 
  \item $\bigcap_{i=1}^n E_i = (\bigcup_{i=1}^n F_i^C)^C \in \mathcal F$.

\end{enumerate}

\end{corollary}

Thus we conclude that $\sigma-$Algebra is a family of subsets closed to any countable number of operations.

\begin{corollary}
  \textbf{(Principle of Inclusion-exclusion)} For all $F_i\in \Sigma,$ we have 
  \begin{equation}
    \nonumber
    \begin{aligned}
      \mu(\bigcup_{i\leq n}F_i & = \Sigma_{i\leq n} \mu(F_i) - \Sigma\Sigma_{i < j\leq n}\mu(F_i\cap F_j) 
    \\ & + \dots +(-1)^{n-1}\mu(F_1\cap F_2\cap\dots \cap F_n).
    \end{aligned}  
  \end{equation}
\end{corollary}

The proof of Corollary 3 has been written in Chen Tong's notes on github.

\subsection{Borel $\sigma$-Algebra}
$$ \mathcal B(S) := \sigma(\text{open subsets of } S) $$

The most common Borel algebra is $\mathcal B := \mathcal B(\mathbb R)$. It consists of the open subsets in $\mathbb R$. But it is hard to find a subset of $\mathbb R$ but not in $\mathcal B$.

\begin{theorem}
Define $\pi(\mathbb R) := \{ (-\infty,x]\mid x\in \mathbb R\}$, then 
$$\mathcal B = \sigma(\pi(\mathbb R)).$$
\end{theorem}

\begin{proof}
  For all $x\in \mathbb R, $. $(-\infty,x] = \bigcap(-\infty,x+1/n)$. Use the property 2, $(-\infty,x]$ is measurable, i.e. $(-\infty,x]\in \mathcal B$.
  
  Then we just need to prove that $\forall a,b\in \mathbb R$, $(a,b)\in \sigma(\pi(\mathbb R))$. First, $(a,b)$ can be represented as $\bigcup_i^\infty (a,b-\epsilon/n]$, where $\epsilon = (b-a)/2$ (in facet it can be arbitrary small). Then we have 
  $$(a,k] = (-\infty,k] \cap (a,\infty) = (-\infty,k] \cap (-\infty,a]^C \in \mathcal B,$$
  thus $(a,b-\epsilon/n] \in \mathcal B$. In this way, $(a,b) = \bigcup_i^\infty (a,b-\epsilon/n] \in \mathcal B$. 
\end{proof}

\subsection{Finite and $\sigma$-finite}
We have a measure space $(S,\Sigma,\mu)$.
\paragraph{Finite} The measure space is finite iff $\mu(S) < \infty$.


\paragraph{$\sigma-$finite} The measure space is $\sigma-$finite iff there exists a sequence $\{S_n\}(S_i\in \Sigma)$,s.t.
$$\mu(S_n) < \infty (\forall n\in \mathbb N) \text{ and } \bigcup S_n = S.$$

\subsection{Minimum $\sigma-$algebra} 
Here  we introduce a signal $\sigma(A)$, which denote the minimum $\sigma$-algebra including $A$.

\section{$\pi$-System}
We have a set S and $\mathcal I \subseteq 2^S$. $(S,\mathcal I)$ is a $\pi$-System iff 
$$I_1,I_2\in\mathcal I \Rightarrow I_1 \cap I_2 \in \mathcal I$$

$\pi$-System is easier for us to research, for it is just closed on $\cap$ while $\sigma$-algebra is closed on both $\cap$ and $\cup$(in my opinion like a group to a ring in abstract algebra).

\begin{theorem}
  Define $\Sigma := \sigma(\mathcal I)$. If there exists two measures $\mu_1,\mu_2$ on $(S,\Sigma)$ satisfy:
  $$\mu_1(S) = \mu_2(S) < \infty$$

and
$$\mu_1(x) = \mu_2(x), \forall x\in \mathcal I,$$

then 
$$\mu_1(x) = \mu_2(x), \forall x\in \Sigma.$$
\end{theorem}

\begin{proof}
  The proof is in Appendix in our textbook, using Dynkin lemma.
\end{proof}
\begin{theorem}
\textbf{(Carath\'eodory Expansion Theorem)} $S$ is a set, and $\Sigma_0$ is an algebra on S($\Sigma_0$ is a $\pi$-system as well). Define $$\Sigma := \sigma(\Sigma_0).$$

 \textbf{(Existence)}  If $\mu_0:\Sigma_0 \mapsto [0,\infty]$ is a mapping satisfies the rule of measure function, then there \underline{exists} a $\mu$ on $\Sigma$ and satisfies 
$$\mu(x) = \mu_0(x), \forall x\in \Sigma_0. $$

\textbf{(Uniqueness)} Futhermore, if $\mu_0(S) < \infty$, then according to Theorem 2, the expansion $\mu$ is unique.

\end{theorem}

\subsection{Lebes\'gue Measure}
Let $S = (0,1].$ Define
$$F = (a_1,b_1] \cup \dots \cup (a_r,b_r] \subseteq S, \text{ where } 0 \leq a_1\leq b_1\leq \dots \leq a_r\leq b_r.$$

And the union of $F$ is defined as $\pi$-System $\Sigma_0$. According to 2.3, we can define a minimum $\sigma$-algebra 
$$\Sigma := \sigma(\Sigma_0) = \mathcal (0,1].$$

Define the measure $\mu_0$ on $\Sigma_0$ as 
$$\mu_0(F) = \Sigma_{k\leq r}(b_k-a_k).$$

We can prove that $\mu_0$ satisfy the premise of theorem 3, thus we conclude that there exists a unique measure $\mu$ on $\Sigma$. We call this measure $\mu$ as Lebegue measure on $(0,1] (\text{Leb.} ([0,1],\mathcal B[0,1])).$

* Note that we can simply define $\mu(\{0\}) = 0$ to obtain 
$$\text{Leb.} ([0,1],\mathcal B[0,1]) = \text{Leb.} ((0,1],\mathcal B(0,1]).$$
\section{Probability space} 
A probability space is a special measure space where $\mu(S) = 1$. Then $\mu$ corresponds to the probability that we're familiar with. This concept will be used in the future.


\end{document}




















