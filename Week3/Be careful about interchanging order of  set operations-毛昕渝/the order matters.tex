\documentclass[12pt]{article}
\usepackage{amssymb}
%\usepackage[UTF8]{ctex}
\usepackage{amsmath}
\usepackage{amsthm}
\usepackage{geometry}
\usepackage{booktabs}
\usepackage{bm}
\usepackage{cite}
%\usepackage{CJK}
\usepackage[many]{tcolorbox}
%\tcbuselibrary{listingsutf8}
%\tcbuselibrary{skins, breakable, theorems, most}
%\geometry{a4paper,bottom = 3cm,left = 3cm, right = 3cm}

%for reference
\usepackage{hyperref}
\usepackage[capitalise]{cleveref}
\crefname{enumi}{}{}

\newtheorem{theorem}{Theorem}
\newtheorem{lemma}[theorem]{Lemma}
\newtheorem{proposition}[theorem]{Proposition}
\newtheorem{fact}[theorem]{Fact}
\newtheorem{definition}[theorem]{Definition}
\newtheorem*{remark}{Remark}

%\newenvironment{proof}{\noindent \textbf{Proof:}}{$\Box$}

%to use newcommand for convenience
\newcommand\field{\mathbb{F}}
\newcommand\Real{\mathbb{R}}
\newcommand\Q{\mathbb{Q}}
\newcommand\Z{\mathbb{Z}}
\newcommand\N{\mathbb{N}}
\newcommand\cc{\mathcal{C}}
\newcommand\bb{\mathcal{B}}
\newcommand\pp{\mathcal{P}}
\newcommand\nn{\mathcal{N}}
\newcommand\ff{\mathcal{F}}
\newcommand\one{\bm{1}}
\newcommand\eps{\varepsilon}

\DeclareMathOperator{\leb}{Leb}   
\DeclareMathOperator{\diff}{d}   

\title{Be careful about interchanging orders of set operations}
\author{Xinyu Mao}
\date{\today}
\begin{document}
\maketitle

\begin{tcolorbox}
\begin{paragraph}{Introduction}
    This is a solution for the last exercise of
    lecture note on March 20.
    \textit{Baire category theorem} is applied here 
    and the result reminds us that 
    we cannot interchange the order of set operations freely.
\end{paragraph}
\end{tcolorbox}

We first recall \text{Baire category theorem}.
A set $A$ is nowhere dense if the interior set of its closure is empty.
\begin{theorem}[Baire category theorem]\label{baire}
In a complete metric space, any set of first category,
namely any countable union of nowhere dense subsets, has empty interior. Equivalently,
any countable intersection of open and dense subsets is still dense.
\end{theorem}   

We now consider the measure space $([0,1],\bb[0,1],\leb)$.
Let $(\eps_k)_{k \in \N}$ be a sequence of strictly positive reals 
such that $\lim_{k \to \infty} \eps_k = 0$. Enumerate 
elements of $\Q \cap [0,1]$ as $(v_n)_{n \in \N}$, and let 
$$
    I_{n,k} := [0,1] \cap [v_n - \frac{\eps_k}{2^n}, v_n + \frac{\eps_k}{2^n}].
$$
Clearly, $\bigcup_n \bigcap_k I_{n,k} = \Q \cap [0,1]$
is a countable set of measure zero. 
And we shall prove:

\begin{proposition}
    The set $T:= \bigcap_k \bigcup_n I_{n,k}$ is an uncountable set of measure zero.
\end{proposition}
\begin{proof}
Let $G_k := \bigcup_n I_{n,k}$, we have 
$\leb(G_k) \leq \sum_{n \in \N} \leb(I_{n,k}) \leq 
\sum_{n \in \N} \frac{\eps_k}{2^{n-1}} = 4\eps_{k}$.

Since $G_{k+1} \subseteq G_k$, we have
$$
\leb(T) = \leb(\bigcap_{k \in \N} G_k) 
= \lim_{k \to \infty} \leb(G_k) \leq \lim_{k \to \infty} 4\eps_{k} = 0,
$$ 
which implies $\leb(T) = 0$.

It remains to show that $T$ is uncountable.
Assume that $T$ is countable for contradiction 
and enumerate it as $(t_n)_{n \in \N}$.
The key observation is that the complement of $G_k$ is nowhere dense.
Indeed, if the closure of $G_k^c$ has non-empty interior,
it must contain some rational number since $\Q$ is dense,
but $\Q \cap G_k^c = \emptyset$.

Note that 
$$
    [0,1] = T \cup T^c = \bigcup_{n \in \N}\{t_n\} \bigcup_{k \in \N} G_k^c. 
$$
Each one of $\{t_n\}$ is nowhere dense without doubt, and so is $G_k^c$,
and we can conclude that $[0,1]$ is has empty interior according to
\cref{baire}, which is ridiculous.
\end{proof}
Since $T$ is uncountable, we have
 $\bigcap_k \bigcup_n I_{n,k} \neq \bigcup_n \bigcap_k I_{n,k}$,
 but they are both of measure zero.


Moreover, consider the sets
$$
    J_{n,k} := [v_n - \eps_k,v_n + \eps_k] \cap [0,1].
$$
One can easily see that $\bigcup_n \bigcap_k J_{n,k} = \Q \cap [0,1]$.
We now take a closer look at $A := \bigcap_k \bigcup_n J_{n,k}$.
Take $x \in [0,1]$. For all $k \in \N$, since $\Q$ is dense, 
there is some $v_i \in \Q$ such that $|x - v_i| \leq \eps_k$,
that is, $x \in \bigcup_n J_{n,k}$. 
It follows that $[0,1] \subseteq A$, which means $A = [0,1]$.
Therefore,
$$
   \leb(A) = \leb(\bigcap_k \bigcup_n J_{n,k}) = 1
   \neq 0 = \leb(\bigcup_n \bigcap_k J_{n,k}).
$$
This again tells us that the orders of operations do matter a lot!
\end{document}