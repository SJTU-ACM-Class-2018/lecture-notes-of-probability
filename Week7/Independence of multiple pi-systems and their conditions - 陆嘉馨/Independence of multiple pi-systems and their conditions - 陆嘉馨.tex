\documentclass{article}
\usepackage{cite}
% \usepackage[UTF8]{ctex}
\usepackage{amssymb}
\usepackage{amsmath}
\usepackage{amsthm}
\usepackage{geometry}
\usepackage{booktabs}
\usepackage{bm}
\usepackage{enumerate}
\usepackage{tcolorbox}
% \CTEXoptions[today=old]
%Some commonly used notations
%\geometry{a4paper,bottom = 3cm,left = 3cm, right = 3cm}

%for reference
\usepackage{hyperref}
\usepackage[capitalise]{cleveref}
\crefname{enumi}{}{}

\newtheorem{theorem}{Theorem}
\newtheorem{lemma}[theorem]{Lemma}
\newtheorem{proposition}[theorem]{Proposition}
\newtheorem{corollary}[theorem]{Corollary}
\newtheorem{fact}[theorem]{Fact}
\newtheorem{definition}[theorem]{Definition}
\newtheorem{remark}[theorem]{Remark}
\newtheorem{question}[theorem]{Question}
\newtheorem{answer}[theorem]{Answer}
\newtheorem{exercise}[theorem]{Exercise}
\newtheorem{example}[theorem]{Example}
%\newenvironment{proof}{\noindent \textbf{Proof:}}{$\Box$}
\newtheorem{observation}[theorem]{Observation}

%to use newcommand for convenience
\newcommand{\mbb}{\mathbb}
\newcommand{\mbf}{\mathbf}
\newcommand{\mbz}{\mathbb{Z}}
\newcommand{\mbn}{\mathbb{N}}
\newcommand{\mbp}{\mathbb{P}}
\newcommand{\mbh}{\mathbb{H}}
\newcommand{\mbq}{\mathbb{Q}}
\newcommand{\vep}{\varepsilon}
\newcommand{\rd}{\mathrm{d}}
\newcommand{\inv}{^{-1}}
\newcommand{\hp}{^\prime}
\newcommand{\mca}{\mathcal{A}}
\newcommand{\mcb}{\mathcal{B}}
\newcommand{\mcc}{\mathcal{C}}
\newcommand{\mcm}{\mathcal{M}}
\newcommand{\mcr}{\mathcal{R}}
\newcommand{\mcf}{\mathcal{F}}
\newcommand{\mcg}{\mathcal{G}}
\newcommand{\mch}{\mathcal{H}}
\newcommand{\mci}{\mathcal{I}}
\newcommand{\mcj}{\mathcal{J}}
\newcommand{\mcp}{\mathcal{P}}
\newcommand{\mfa}{\mathfrak{A}}
\newcommand{\mfb}{\mathfrak{B}}
\newcommand{\mfc}{\mathfrak{C}}
\newcommand{\mfi}{\mathfrak{I}}
\newcommand{\mfp}{\mathfrak{P}}
\newcommand{\Iff}{\mbox{iff }}
\newcommand{\AND}{\mbox{ and }}

%this is how we define operators.
\DeclareMathOperator{\rank}{rank} % rank

\title{Independence of multiple $\pi$-systems and their conditions}
\author{Lu Jiaxin\\
Student ID: 518030910412}
\date{\today}

\begin{document}
    \maketitle

Here is an lemma in \textit{Probability with Martingales}.

This lemma shows us the independence of two $\pi$-systems and its approach of proof is also used below.

\begin{tcolorbox}
\begin{lemma}[LEMMA 4.2(a)]\label{lemma4.2}
    Suppose that $\mcg$ and $\mch$ are sub-$\sigma$-algebras of $\mcf$ and that $\mci$ and $\mcj$ are $\pi$-systems with
    \[
        \sigma(\mci) = \mcg, \quad \sigma(\mcj) = \mch
    \]
    Then $\mcg$ and $\mch$ are independent if and only if $\mci$ and $\mcj$ are independent in that
    \[
        P(I\cap J) = P(I) P(J), \quad I\in\mci, J\in\mcj.        
    \]
\end{lemma}
\end{tcolorbox}

\begin{proof}
    Suppose that $\mci$ and $\mcj$ are independent. For fixed $I$ in $\mci$,
    \[
        H\mapsto P(I\cap H) \AND H\mapsto P(I)P(H)
    \]
    are measures(since they are maps $\mcf \to [0, \infty]$ on $(\Omega, \mcf)$ and by definition they are measures) on $(\Omega, \mch)$ have the same total mass $P(I)$, and agree on $\mcj$. By the uniqueness of extension(see in the same book \textbf{Lemma 1.6}), they therefore agree on $\sigma(\mcj) = \mch$. Hence,
    \[
        P(I\cap H) = P(I) P(H), \quad I\in\mci, H\in\mch.
    \]
    
    Thus, for fixed $H$ in $\mch$, the measures
    \[
        G\mapsto P(G\cap H) \AND G\mapsto P(G)P(H)
    \]
    on $(\Omega, \mcg)$ have the same total mass $P(H)$, and agree on $\mci$. They therefore agree on $\sigma(\mci) = \mcg$.
    
    Thus we finish our proof.
\end{proof}


Now consider the case where there are three $\pi$-systems. This is the case in \textbf{Exercise 4.1}.

This exercise requires us to prove the independence of three $\pi$-systems. We can prove it with the approach above. But we'll have to face a problem that to complete the proof, we must add a condition.

I'll explain the reason in the remark below.

\begin{tcolorbox}
\begin{exercise}[E4.1]\label{E4.1}
    Let $(\Omega, \mcf, P)$ be a probability triple. Let $\mci_1, \mci_2 and \mci_3$ be three $\pi$-systems on $\Omega$ such that, for $k=1,2,3,$
    \begin{align*}
        \mci_k \subseteq \mcf \\
    \end{align*}
    and
    \begin{align*} \label{cond1}
        \Omega \in \mci_k
    \end{align*}
    Prove that if
    \[
        P(I_1 \cap I_2 \cap I_3) = P(I_1) P(I_2) P(I_3)
    \]
    whenever $I_k\in \mci_k \, (k = 1,2,3)$ then $\sigma(\mci_1), \sigma(\mci_2), \sigma(\mci_3)$ are independent.
\end{exercise}
\end{tcolorbox}

\begin{proof}
    Let $\sigma(\mci_3) = \mcj_3$. Fix $I_1 \in \mci_1$ and $I_2 \in \mci_2$. Consider the maps
    \[
        J_3 \mapsto P(I_1\cap I_2 \cap J_3) \AND J_3\mapsto P(I_1)P(I_2)P(I_3),
    \]
    for $J_3 \in \sigma(\mci_3)$. We can verify that these are measures on the measure space $(\Omega, \sigma(\mci_3))$.

    Note that since $\Omega \in \mci_3$, we derive
    \begin{align}\label{eq2}
        P(I_1 \cap I_2) = P(I_1 \cap I_2 \cap \Omega) = P(I_1) P(I_2) P(\Omega) = P(I_1) P(I_2),        
    \end{align}
    so these measures both have a total mass of 
    \[
        P(I_1 \cap I_2) \AND P(I_1)P(I_2),
    \]
    which are exactly the same, and agree on $\mci_3$.

    By the uniqueness of extension(see in the same book \textbf{Lemma 1.6}), they therefore agree on $\sigma(\mci_3) = \mcj_3$.
    Using this approach on $k = 1,2$ we naturally have $\sigma(\mci_1), \sigma(\mci_2), \sigma(\mci_3)$ are independent.
\end{proof}

\begin{remark}
    Although the approachs used in both proofs look the exactly the same, the conditions of two statements are different.
    
    \cref{E4.1} requires the condition $\Omega \in \mci_k$ while \cref{lemma4.2} doesn't. Here's my thought.
    
    In the proof of \cref{lemma4.2}, the sentence \textit{'have the same total mass $P(I)$'} comes naturally since when there are two $\pi$-systems,
    \[
        P(I \cap \Omega) = P(I)P(\Omega)
    \]
    holds vacuously. So including the hypothesis would be unnecessary.
    
    While in the proof of \cref{E4.1}, we have to use \cref{eq2} to prove that
    \[
        P(I_1 \cap I_2) = P(I_1) P(I_2)
    \]
    under the condition $\Omega \in \mci_k$. Because if $\Omega \notin \mci_k$, we can't derive $P(I_1 \cap I_2) = P(I_1) P(I_2)$ from $P(I_1 \cap I_2 \cap I_3) = P(I_1) P(I_2) P(I_3)$.
    
    This condition is also required if we want to extend this theorem to arbitary $n\in \mbn (n \geq 3)$ $\pi$-systems. Otherwise, we won't be able to apply the uniqueness of extension(see in the same book \textbf{Lemma 1.6}) to verify two measures we create agree on $\sigma(\mci_k)$.
    
\end{remark}
\end{document}