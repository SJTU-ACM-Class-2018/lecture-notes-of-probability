\documentclass[a4paper, linespread=1.5]{article}
%\usepackage[UTF8]{ctex}
\usepackage{xeCJK, geometry, amsmath, amssymb, amsthm}
\usepackage{graphicx, keyval}
\usepackage[dvipsnames,svgnames,x11names]{xcolor}
\usepackage{float, ifthen, calc, ifplatform, fancyvrb}
\usepackage{minted, hyperref, enumerate, multicol}
\usepackage[all]{xy}
\usepackage{ulem}
%\usepackage{epstopdf}
\usepackage{mathrsfs}
\usepackage{cancel}
\usepackage{algorithm, algorithmic}

%\setlength{\parskip}{0.2\baselineskip}
%\setlength{\parindent}{2em}
%\geometry{left=2.7cm,right=2.7cm,top=2.7cm,bottom=2.7cm}


\newtheorem{theorem}{Theorem}
\newtheorem{proposition}[theorem]{Proposition}
\newtheorem{lemma}[theorem]{Lemma}
\newtheorem{corollary}[theorem]{Corollary}
\newtheorem{definition}[theorem]{Definition}
\newtheorem{exercise}[theorem]{Exercise}
\newtheorem{property}[theorem]{Property}

\newtheorem{innercustom}{\customname}
\providecommand{\customname}{}
\newcommand{\newcustomtheorem}[2]{
    \newenvironment{#1}[1]
    {
        \renewcommand\customname{#2}
        \renewcommand\theinnercustom{##1}
        \innercustom
    }
    {\endinnercustom}
}
\newcustomtheorem{customthm}{Theorem}
\newcustomtheorem{customprop}{Proposition}
\newcustomtheorem{customlemma}{Lemma}
\newcustomtheorem{customcorollary}{Corollary}
\newcustomtheorem{customdef}{Definition}
\newcustomtheorem{customex}{Exercise}
\newcustomtheorem{customremark}{Remark}

\newcommand{\Natural}{\mathbb{N}}
\newcommand{\Real}{\mathbb{R}}
\newcommand{\BorelSet}{\mathcal{B}}
\newcommand{\addbigcup}{\bigcup{\kern-1.12em{+}}\kern0.3em}
\newcommand{\nth}[1]{#1\textsuperscript{th}}
\newcommand{\IndicatorFunc}[1]{\mathbf{1}_{#1}}

\begin{document}
    \title{Integral of Positive Simple Functions($SF^+$)}
    \author{赖睿航 518030910422}
    \date{\today}
    \maketitle
    
    Let $(S, \Sigma, \mu)$ be a measure space. A function $f \in (m\Sigma)^+$ is called {\it simple} provided $f$ can be written as a finite sum
    $$
    f = \sum_{k = 1}^{m} a_k \IndicatorFunc{A_k},
    $$
    where $a_k \in [0, \infty]$, $A_k \in \Sigma$. We write $f \in SF^+$. Then we define the \textit{Lebesgue integral} of $f$ as
    $$
    \mu(f) = \sum_{k = 1}^{m} a_k \mu(A_k) \leqslant \infty.
    $$
    
    \bigskip
    
    \section{Lebesgue Integral of Simple Function is {\it Well-Defined}}
        We first show that the $\mu(f)$ is well-defined.
        \begin{property}[Well-definition]
            Let $\sum_{i = 1}^n a_i\IndicatorFunc{A_i} = \sum_{j = 1}^m b_j\IndicatorFunc{B_j}$ be two representations of the same simple function $f \in SF^+$ with $A_i, B_j \in [0, \infty]$. Then
            $$
            \sum_{i = 1}^n a_i\mu(A_i) = \sum_{j = 1}^{m} b_j\mu(B_j).
            $$
        \end{property}

        \begin{proof}
            Note that for $p, q \in [n]$ with $p \neq q$, $A_p$ and $A_q$ may have \textit{non-empty intersection}, which is not what we want. So we construct some {\it pairwise disjoint} sets from $(A_i)$: let $(r_{t_1}r_{t_2}\ldots r_{t_n})_2$ be the binary representation of $t \in \{0, 1, \ldots 2^n - 1\}$, where $r_{t_k}$ is either $0$ or $1$. Then we define set $C_t$ as
            $$
            C_t := \bigcap_{k = 1}^n R_{t_k},\ \textrm{where } R_{t_k} = \left\{
            \begin{aligned}
            A_k, \textrm{ if } r_{t_k} = 1 \\
            A_k^c, \textrm{ if } r_{t_k} = 0
            \end{aligned}
            \right.
            $$
            For example let $n = 3$, and then we have
            \begin{align*}
                C_0 &= A_1^c \cap A_2^c \cap A_3^c,\ C_1 = A_1 \cap A_2^c \cap A_3^c \\
                C_2 &= A_1^c \cap A_2 \cap A_3^c,\ C_3 = A_1 \cap A_2 \cap A_3^c \\
                C_4 &= A_1^c \cap A_2^c \cap A_3,\ C_5 = A_1 \cap A_2^c \cap A_3 \\
                C_6 &= A_1^c \cap A_2 \cap A_3,\ C_7 = A_1 \cap A_2 \cap A_3.
            \end{align*}
            Obviously the sets in $(C_t)$ are pairwise disjoint. Then let
            $$
            c_i := \sum_{j = 1}^n [A_j \cap C_i \neq \emptyset]a_j
            $$
            where $[x] = 1$ provided $x$ is true, and $[x] = 0$ otherwise. Hence we have
            $$
            f = \sum_{i = 1}^n a_i\IndicatorFunc{A_i} = \sum_{i = 0}^{2^n - 1} c_i\IndicatorFunc{C_i},\ \sum_{i = 1}^n a_i\mu(A_i) = \sum_{i = 0}^{2^n - 1} c_i\mu(C_i).
            $$
            Note that for some $i \in \{0, 1, \ldots 2^n - 1\}$, $c_i$ may be $0$, which means $c_i\IndicatorFunc{C_i} = 0$. We remove such terms from the representation of $f$ and finally we will obtain
            $$
            f = \sum_{i = 1}^n a_i\IndicatorFunc{A_i} = \sum_{i = 0}^N c_i\IndicatorFunc{C_i},\ \sum_{i = 1}^n a_i\mu(A_i) = \sum_{i = 0}^N c_i\mu(C_i)
            $$
            where $N \leqslant 2^n - 1$, $c_i > 0$, and sets in $(C_i)$ are pairwise disjoint. Similarly we also have
            $$
            f = \sum_{i = 1}^m b_i\IndicatorFunc{B_i} = \sum_{i = 0}^M d_i\IndicatorFunc{D_i},\ \sum_{j = 1}^m b_i\mu(B_i) = \sum_{j = 0}^M d_j\mu(D_j).
            $$
            \textbf{So we only need to show that}
            $$
            \sum_{i = 0}^N c_i\mu(C_i) = \sum_{j = 0}^M d_j\mu(D_j).
            $$
            with factors $c_i, d_j$ being positive and sets in both $(C_i), (D_j)$ being pairwise disjoint.
            \bigskip
            
            Claim that $\bigsqcup_{i = 0}^N C_i = \bigsqcup_{j = 0}^M D_j$: if $x \in \bigsqcup_{i = 0}^N C_i$, then $f(x) > 0$ since $c_i > 0$ always holds, which means $x \in \bigsqcup_{j = 0}^M D_j$ as well, and vice versa.
            
            Then for every $C_i$, we have $C_i \subseteq \bigsqcup_{i = 0}^N C_i = \bigsqcup_{j = 0}^M D_j$, so $C_i = \bigsqcup_{j = 0}^M (D_j \cap C_i)$. Then we have
            \begin{align*}
            \sum_{i = 0}^N c_i\IndicatorFunc{C_i} &= \sum_{i = 0}^N c_i\IndicatorFunc{\bigsqcup_{j = 0}^M(D_j \cap C_i)} \\
            &= \sum_{i = 0}^N c_i \sum_{j = 0}^M \IndicatorFunc{D_j \cap C_i} \\
            &= \sum_{i = 0}^N \sum_{j = 0}^M c_i\IndicatorFunc{C_i \cap D_j}.
            \end{align*}
            Similarly, we also have
            $$
            \sum_{j = 0}^M d_j\IndicatorFunc{D_j} = \sum_{i = 0}^N \sum_{j = 0}^M d_j\IndicatorFunc{C_i \cap D_j}.
            $$
            So for any $i, j$, if $C_i \cap D_j \neq \emptyset$, then $c_i = d_j$ must be true.
            
            We perform the same operations to $\sum_{i = 0}^N c_i\mu(C_i)$ and $\sum_{j = 0}^M d_i\mu(D_i)$:
            \begin{align*}
            \sum_{i = 0}^N c_i\mu(C_i) &= \sum_{i = 0}^N c_i \mu(\bigsqcup_{j = 0}^M (D_j \cap C_i)) \\
            &= \sum_{i = 0}^N c_i \sum_{j = 0}^M \mu(D_j \cap C_i) \\
            &= \sum_{i = 0}^N \sum_{j = 0}^M c_i \mu(C_i \cap D_j), \\
            \sum_{j = 0}^M d_j\mu(D_j) &= \sum_{i = 0}^N \sum_{j = 0}^M d_j \mu(C_i \cap D_j).
            \end{align*}
            Observe that if $C_i \cap D_j \neq \emptyset$, then we know that $c_i = d_j$. Otherwise if $C_i \cap D_j = \emptyset$, then $\mu(C_i \cap D_j) = 0$. So for any $i, j$,
            $$
            c_i \mu(C_i \cap D_j) = d_j \mu(C_i \cap D_j).
            $$
            Thus we have $\sum_{i = 0}^N \sum_{j = 0}^M c_i \mu(C_i \cap D_j) = \sum_{i = 0}^N \sum_{j = 0}^M d_j \mu(C_i \cap D_j)$, i.e.,
            $$
            \sum_{i = 0}^N c_i\mu(C_i) = \sum_{j = 0}^M d_j\mu(D_j).
            $$
        \end{proof}
\end{document}
