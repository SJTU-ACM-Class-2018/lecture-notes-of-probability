% This is a template for lecture notes.
\documentclass[12pt]{article}
\usepackage{amssymb}
\usepackage[UTF8]{ctex}
\usepackage{amsmath}
\usepackage{amsthm}
\usepackage{geometry}
\usepackage{booktabs}
\usepackage{bm}
\usepackage{cite}
%\usepackage{CJK}
\usepackage[many]{tcolorbox}
%\tcbuselibrary{listingsutf8}
%\tcbuselibrary{skins, breakable, theorems, most}
%\geometry{a4paper,bottom = 3cm,left = 3cm, right = 3cm}
\CTEXoptions[today=old]
%for reference
\usepackage{hyperref}
\usepackage[capitalise]{cleveref}
\crefname{enumi}{}{}

\newtheoremstyle{mythm}{1.5ex plus 1ex minus .2ex}{1.5ex plus 1ex minus .2ex} 
    {}{\parindent}{\bfseries}{}{1em}{} 
\theoremstyle{mythm}
\newtheorem{theorem}{Theorem}
\newtheorem{lemma}[theorem]{Lemma}
\newtheorem{corollary}[theorem]{Corollary}
\newtheorem{fact}[theorem]{Fact}
\newtheorem{definition}[theorem]{Definition}
\newtheorem*{remark}{Remark}

%\newenvironment{proof}{\noindent \textbf{Proof:}}{$\Box$}

%to use newcommand for convenience
\newcommand\field{\mathbb{F}}
\newcommand\Real{\mathbb{R}}
\newcommand\Q{\mathbb{Q}}
\newcommand\Z{\mathbb{Z}}
\newcommand\complex{\mathbb{C}}
\newcommand\cc{\mathcal{C}}
\newcommand\uu{\mathcal{U}}
\newcommand\pp{\mathcal{P}}
\newcommand\ff{\mathcal{F}}
\renewcommand\refname{Reference}
\renewcommand{\proofname}{Proof}
\DeclareMathOperator{\range}{range}   

\title{Proof of a Simple Lemma}
\author{马浩博 518030910428}
\date{\today}
\begin{document}
\maketitle

This is an easy exercise arranged in class. Nobody did it last week, so I give out a simple proof.

\begin{lemma}
	 Suppose that $h$ is a Borel measurable function from $R$ to $R$. Then
	 $$h(X) \in \mathcal{L}^1(\Omega,\mathcal{F},P) \ if \ and \ only \ if \ h \in \mathcal{L}^1(R,\mathcal{B},\Lambda_X )$$
	 and then
	 $$Eh(X) = \Lambda_X(h) = \int_{R} h(x) \Lambda_X\, (dx).$$
\end{lemma}

\begin{proof}
	Proof is simple cause we can just feed everything into the standard machine.
	First if $h = I_B (B \in \mathcal{B})$, by the definition of $\Lambda_X$ we know that $$Eh(X) = \Lambda_X(h) = \int_{R} h(x) \Lambda_X\, (dx).$$
	Then use linearity to tell that the conclusion is still true when $h$ is a simple function on $(R,\mathcal{B})$. 
	Next, when $h$ is a non-negative function, by MON we can also get that 
	$$Eh(X) = \Lambda_X(h) = \int_{R} h(x) \Lambda_X\, (dx).$$
	Use linearity again and we finish the proof.
\end{proof}

\end{document}