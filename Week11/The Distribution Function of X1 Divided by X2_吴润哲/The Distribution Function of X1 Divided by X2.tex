\documentclass[12pt]{article}

%\usepackage[UTF8]{ctex}
\usepackage{geometry}
\usepackage{amsthm}
\usepackage{amsmath}
\usepackage{amssymb}
\usepackage{mathtools}
\usepackage{enumerate}
\usepackage{hyperref} 
\usepackage{tcolorbox}

\geometry{a4paper, left = 2cm, right = 2cm, top = 2cm}

\newcommand\problem[1]{\section*{Problem #1}}

\newcommand\bE{\mathbb{E}}
\newcommand\bF{\mathbb{F}}
\newcommand\bN{\mathbb{N}}
\newcommand\bZ{\mathbb{Z}}
\newcommand\bQ{\mathbb{Q}}
\newcommand\bR{\mathbb{R}}
\newcommand\fC{\mathbf{C}}
\newcommand\fF{\mathbf{F}}
\newcommand\fN{\mathbf{N}}
\newcommand\fQ{\mathbf{Q}}
\newcommand\fR{\mathbf{R}}
\newcommand\fZ{\mathbf{Z}}
\newcommand\cF{\mathcal{F}}
\newcommand\cU{\mathcal{U}}
\newcommand\N{\mathcal{N}}

\newcommand\pro{\mathbf{P}}
\newcommand\expt{\mathbf{E}}
\newcommand\ce{\coloneqq}
\newcommand\lproof{\item ``$\Leftarrow$'' :}
\newcommand\rproof{\item ``$\Rightarrow$'' :}

\newcommand{\leb}{\text{Leb}}
\newcommand*{\dif}{\mathop{}\!\mathrm{d}}
\newcommand{\ord}{\text{ord}}
\newcommand{\floor}[1]{\lfloor {#1}\rfloor}
\newcommand{\ind}[1]{\mathbf{1}_{#1}}
\newcommand{\ms}{\rm{m}\Sigma}

\newtheorem{claim}{Claim}
\newtheorem{definition}{Definition}
\newtheorem{lemma}{Lemma}
\newtheorem{theorem}{Theorem}
\newtheorem{corollary}{Corollary}
\newtheorem{remark}{Remark}
\newtheorem*{exercise}{Exercise}

\title{The Distribution Function of X1 Divided by X2}
\author{WU Runzhe\\
	Student ID : 518030910432}
\date{}

\setlength{\parskip}{1em}

\begin{document}
	\maketitle \large
	
	\begin{tcolorbox}
		\begin{exercise}
			Given $(X_1,X_2)^T\sim \N(\mu,V)$ where 
			$$
			\mu = 
			\left(
			\begin{matrix}
			0\\
			0
			\end{matrix}
			\right),
			V=
			\left(
			\begin{matrix}
			1&0\\
			0&1
			\end{matrix}
			\right)
			$$
			then what's the distribution function of $X_1\over X_2$?
		\end{exercise}
	\end{tcolorbox}
	
	Let's first recall the definition of multivariate normal distribution.
	
	\begin{definition}\label{d1}
		A random vector $\textbf{X}=(X_1,X_2,\dots,X_n)^T$ is called a normal random vector, which means $\textbf{X}\sim \N(\mu,V)$ for some $\mu,V$, if it has the probability density function for $x\in \bR^n$
		$$f(x)=\frac{\sqrt{|V^{-1}|}}{(2\pi)^{n\over 2}}e^{-\frac{1}{2}(x-\mu)^TV^{-1}(x-\mu)}$$
	\end{definition}
	
	It remains to check some important properties like $\int_{x\in\bR^n}f(x)\dif x=1$ and so on. However, the following definition is equivalent to the above, but the proof of equivalence seems to be complicated.
	
	\begin{definition}\label{d2}
		A random vector $\textbf{X}=(X_1,X_2,\dots,X_n)^T$ is called a normal random vector if  $\sum_{i=1}^{n}r_i X_i$ is normal for any $r_i\in \bR$.
	\end{definition}

	Now let's get back to the exercise. Consider the following general theorem.
	
	\begin{theorem}\label{t1}
		Let $(X,Y)^T$ be a random vector with probability density function $f(x,y)$. Then the probability density function of $\frac{Y}{X}$ is given by
		$$f_{Y/X}(z)=\int_{-\infty}^{+\infty}|x|f(x,xz)\dif x.$$
	\end{theorem}

	\begin{proof}[Proof of theorem \ref{t1}]
		Let $G=\{(x,y):\frac{y}{x}\le z\}$ for some $z$. And then the distribution function of $\frac{Y}{X}$ is
		\begin{align*}
			F_{Y/X}(z)&=P\{Y/X\le z\}=\iint_{G} f(x,y) \dif x\dif y\\
			&=(\int_{y/x\le z,x<0}+\int_{y/x\le x,x>0}) f(x,y)\dif x\dif y\\
			&=\int_{-\infty}^0\left[\int_{xz}^{\infty}f(x,y)\dif y\right]\dif x+\int_{0}^{\infty}\left[\int_{-\infty}^{xz}f(x,y)\dif y\right]\dif x\\
			&=\int_{-\infty}^{z}\left[\int_{-\infty}^{\infty}|x|f(x,xw)\right]\dif w
		\end{align*}
		Since typing formulas in \LaTeX{} is tough and despairing, I omit some intermediate step and give the final expression directly. By the definition of pdf, we conclude $$f_{Y/X}(z)=\int_{-\infty}^{+\infty}|x|f(x,xz)\dif x.$$
	\end{proof}
	
	Suppose $(X_1,X_2)^T\sim \N(\mu,V)$ where 
	$$	\mu = 	\left(	\begin{matrix}	0\\	0	\end{matrix}	\right),
	V=	\left(	\begin{matrix}	1&0\\	0&1	\end{matrix}	\right)	.$$
	
	By definition \ref{d1}, it has probability density function 
	$$f(x_1,x_2)=\frac{1}{2\pi}e^{-\frac{x_1^2+x_2^2}{2}}$$
	
	Applying theorem \ref{t1}, all we need to do is computing 
	
	\begin{align*}
		f_{X_1/X_2}(z)&=\int_{-\infty}^{+\infty}|x|f(xz,x)\dif x\\
		&=2\int_{0}^{+\infty}xf(xz,x)\dif x\\
		&=\int_{0}^{+\infty}f(xz,x)\dif x^2\\
		&=\frac{1}{2\pi}\int_{0}^{+\infty}e^{-\frac{x^2(1+z^2)}{2}}\dif x^2\\
		&=-\frac{1}{\pi(1+z^2)} \left.e^{-\frac{t(1+z^2)}{2}}\right|^{+\infty}_0\\
		&=\frac{1}{\pi(1+z^2)}
	\end{align*}
	
	Thus, the distribution function of $\frac{X_1}{X_2}$ is
	
	$$F_{X_1/X_2}(x)=\int_{-\infty}^{x}\frac{1}{\pi(1+z^2)}\dif z=\frac{\arctan(x)}{\pi}+\frac{1}{2}$$

	
\end{document}
