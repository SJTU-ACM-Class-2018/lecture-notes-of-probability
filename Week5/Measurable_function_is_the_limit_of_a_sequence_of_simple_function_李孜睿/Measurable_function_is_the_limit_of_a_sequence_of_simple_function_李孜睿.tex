% This is a template for lecture notes.
\documentclass{article}
\usepackage[UTF8]{ctex}
\usepackage{amssymb}
\usepackage{amsmath}
\usepackage{amsthm}
\usepackage{geometry}
\usepackage{booktabs}
\usepackage{bm}
\usepackage{tcolorbox}
\CTEXoptions[today=old]
%Some commonly used notations
%\geometry{a4paper,bottom = 3cm,left = 3cm, right = 3cm}

%for reference
\usepackage{hyperref}
\usepackage[capitalise]{cleveref}
\crefname{enumi}{}{}

\newtheorem{theorem}{Theorem}
\newtheorem{lemma}[theorem]{Lemma}
\newtheorem{proposition}[theorem]{Proposition}
\newtheorem{corollary}[theorem]{Corollary}
\newtheorem{fact}[theorem]{Fact}
\newtheorem{definition}[theorem]{Definition}
\newtheorem{remark}[theorem]{Remark}
\newtheorem{question}[theorem]{Question}
\newtheorem{answer}[theorem]{Answer}
\newtheorem{exercise}[theorem]{Exercise}
\newtheorem{example}[theorem]{Example}
%\newenvironment{proof}{\noindent \textbf{Proof:}}{$\Box$}
\newtheorem{observation}[theorem]{Observation}

%to use newcommand for convenience
\newcommand\field{\mathbb{F}}
\newcommand\Real{\mathbb{R}}
\newcommand\Q{\mathbb{Q}}
\newcommand\Z{\mathbb{Z}}
\newcommand\complex{\mathbb{C}}
\newenvironment{myproof}{\ignorespaces\paragraph{Proof:}}{\hfill $\square$\par\noindent}
%this is how we define operators.
\DeclareMathOperator{\rank}{rank} % rank

\title{Measurable function is the limit of a sequence of simple function}
\author{李孜睿 518030910424}
\date{\today}


\begin{document}
	\maketitle
	Define the dyadic function $d_n\in \mathbb{R}^{\mathbb{R}}$
	$$d_n=\sum_{k=1}^{n2^n}\frac{k-1}{2^n}\boldsymbol{1}_{[\frac{k-1}{2^n}, \frac{k}{2^n})}+n\boldsymbol{1}_{[n,\infty)}$$
	
	$f\in \mathbb{R}^S$, define $f^+=\max(f, 0)$, $f^-=\max(-f, 0)$.
	
	1. Take $f\in m\Sigma$. For each $n\in \mathbb{N}$, show that $f_n\doteq d_n\circ f^+ - d_n\circ f^-$ is a simple function with respect to $(S, \Sigma)$.
	\begin{myproof}
		If $f(s)\geq0$, $f^+(s)=f(s)\geq0$ and $f^-(s)=0$. So $f_n(s)=d_n\circ f(s)$.
				
		$$f_n(s)=\left\{
		\begin{array}{lr}
		\frac{k-1}{2^n},&f(s)\in [\frac{k-1}{2^n}, \frac{k}{2^n}), k\in[1, n2^n]\cap\mathbb{Z}\\
		n,&f(s)\in [n,\infty)
		\end{array}\right.
		$$
		
		f is a measurable function, $\{[\frac{k-1}{2^n}, \frac{k}{2^n}), [n,\infty)\}\subseteq\mathcal{B}$($\mathcal{B}$ is Borel set).
		
		We can induce $\{f^{-1}([\frac{k-1}{2^n}, \frac{k}{2^n})), f^{-1}([n,\infty))\}\subseteq S$. Then change the form of $f_n(s)$:
		
		
		$$f_n(s)=\left\{
		\begin{array}{lr}
		\frac{k-1}{2^n},&s\in f^{-1}([\frac{k-1}{2^n}, \frac{k}{2^n})), k\in[1, n2^n]\cap\mathbb{Z}\\
		n,&s\in f^{-1}([n,\infty))
		\end{array}\right.
		$$
		
		So $f_n(s)=\sum_{k=1}^{n2^n}\frac{k-1}{2^n}\boldsymbol{1}_{f^{-1}([\frac{k-1}{2^n}, \frac{k}{2^n}))}+n\boldsymbol{1}_{f^{-1}([n,\infty))}$ for $f(s)>0$.

		If $f(s)\leq0$, $f^+(s)=0$ and $f^-(s)=-f(s)\geq0$. So $f_n(s)=-d_n\circ f^-(s)=-d_n\circ -f(s)$. In this case:
		
		$$
		\begin{aligned}
		f_n(s)=&\left\{
		\begin{array}{lr}
		-\frac{k-1}{2^n},&-f(s)\in [\frac{k-1}{2^n}, \frac{k}{2^n}), k\in[1, n2^n]\cap\mathbb{Z}\\
		-n,&-f(s)\in [n,\infty)
		\end{array}\right.\\
		=&\left\{
		\begin{array}{lr}
		\frac{k+1}{2^n},&f(s)\in (\frac{k}{2^n}, \frac{k+1}{2^n}], k\in[-n2^n, -1]\cap\mathbb{Z}\\
		-n,&f(s)\in (-\infty,-n]
		\end{array}\right.\\
		=&\left\{
		\begin{array}{lr}
		\frac{k+1}{2^n},&s\in f^{-1}((\frac{k}{2^n}, \frac{k+1}{2^n}]), k\in[-n2^n, -1]\cap\mathbb{Z}\\
		-n,&s\in f^{-1}((-\infty,-n])
		\end{array}\right.
		\end{aligned}
		$$
		
		
		Above all, we can conclude that
		
		$$
		f_n(s)=\left\{
		\begin{array}{lr}
		n,&s\in f^{-1}([n,\infty))\\
		\frac{k-1}{2^n},&s\in f^{-1}([\frac{k-1}{2^n}, \frac{k}{2^n})), k\in[1, n2^n]\cap\mathbb{Z}\\
		\frac{k+1}{2^n},&s\in f^{-1}((\frac{k}{2^n}, \frac{k+1}{2^n}]), k\in[-n2^n, -1]\cap\mathbb{Z}\\
		-n,&s\in f^{-1}((-\infty,-n])
		\end{array}\right.
		$$
		
		Notice that when k = 1 or -1, $f_n(s)=0$. Above all, we can rewrite $f_n$ to a simple function.
		$$
		f_n=\sum_{i=1}^{n2^n}\frac{k-1}{2^n}\boldsymbol{1}_{f^{-1}([\frac{k-1}{2^n}, \frac{k}{2^n}))}+\sum_{i=-n2^n}^{-1}\frac{k+1}{2^n}\boldsymbol{1}_{f^{-1}((\frac{k}{2^n}, \frac{k+1}{2^n}])}+n\boldsymbol{1}_{[n,\infty)}-n\boldsymbol{1}_{(-\infty,-n]}
		$$
	\end{myproof}

	2. $f_n\uparrow f$
	
	\begin{myproof}
		We need to prove for every $s\in S$, $\lim_{n\rightarrow\infty}f_n(s)=f(s)$.
		
		Assume $f(s)\geq0$. There exists an n that $f(s)<n$. Also, there exists a k that $f(s)\in[\frac{k-1}{2^n},\frac{k}{2^n})$.
		
		As $f(s)\in[\frac{k-1}{2^n},\frac{k}{2^n})$, we get $f_n(s)=\frac{k-1}{2^n}$ from the deduction above.
		
		We have $0\leq f(s)-f_n(s)<\frac{k}{2^n}-\frac{k-1}{2^n}=\frac{1}{2^n}$ and 		$\lim_{n\rightarrow\infty}\frac{1}{2^n}=0$.
		
		By Squeeze Theorem, $\lim_{n\rightarrow\infty}f(s)-f_n(s)=0\Rightarrow\lim_{n\rightarrow\infty}f_n(s)=f(s)$
		
		The proof of the other case $f(s)<0$ is similar.
	\end{myproof}
\end{document}