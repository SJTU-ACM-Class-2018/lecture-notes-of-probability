\documentclass{article}
%\usepackage[UTF8]{ctex}
\usepackage[utf8]{inputenc}
\usepackage{amssymb}
\usepackage{amsmath}
\usepackage{amsthm}
\usepackage{geometry}
\usepackage{booktabs}
\usepackage{bm}
\usepackage{tcolorbox}
\usepackage{setspace}
\usepackage{unicode-math}

\renewcommand{\baselinestretch}{1.3}

\newtheorem{theorem}{Theorem}
\newtheorem{lemma}[theorem]{Lemma}
\newtheorem{claim}[theorem]{Claim}
\newtheorem{proposition}[theorem]{Proposition}
\newtheorem{corollary}[theorem]{Corollary}
\newtheorem{fact}[theorem]{Fact}
\newtheorem{definition}[theorem]{Definition}
\newtheorem{remark}[theorem]{Remark}
\newtheorem{question}[theorem]{Question}
\newtheorem{answer}[theorem]{Answer}
\newtheorem{exercise}[theorem]{Exercise}
\newtheorem{example}[theorem]{Example}
\newtheorem{observation}[theorem]{Observation}
\newtheorem{problem}[theorem]{problem}
\newtheorem*{solution}{Solution}

\title{$\sigma$-algebra Generated By Stopping Time}
\author{Guo Linsong 518030910419}
\date{\today}

\begin{document}

\maketitle


\begin{tcolorbox}
    \begin{definition}
        Let $(\Omega, \mathcal{F}, \mathbb{P})$ be a probability space with a filtration $(\mathcal{F}_n:n\geq 0)$, $X =(X_n:n\geq 0)$ be a process adapted to $(\mathcal{F}_n)$ and $T$ be a stopping time. We define $\sigma$-algebra generated by $T$:
        $$\mathcal{F}_{T}=\sigma\{A\in \mathcal{F},A\cap\{T\leq n\}\in\mathcal{F}_n,\forall n\}$$ 
   \end{definition}
\end{tcolorbox}

Next we prove that $\mathcal{F}_T$ is a $\sigma$-algebra.
\begin{proof}
\begin{enumerate}
    \item  $\emptyset \cap \{T\leq n\}=\emptyset\in\mathcal{F}_n$ implies $\emptyset\in\mathcal{F}_T$.
    \item$\Omega\cap\{T\leq n\}=\{T\leq n\}\in\mathcal{F}_n$ implies $\Omega\in\mathcal{F}_T$.
    \item  If $A\in\mathcal{F}_T$, then $A\cap\{T\leq n\}\in\mathcal{F}_n$ for every $n$. \\
    Thus $A^c\cap\{T\leq n\}=(A^c\cap\{T\leq n\})^c\cap\{T\leq n\}\in\mathcal{F}_n$ implies $A^c\in\mathcal{F}_T$.
    \item If $A_i\in\mathcal{F}_T$ for every $i$, then $(A_i\cap\{T\leq n\}) \in\mathcal{F}_n$ for every $n$.\\
    Thus $(\bigcup_i A_i)\cap\{T\leq n\}=\bigcup_i (A_i\cap\{T\leq n\})\in\mathcal{F}_n$ implies $\bigcup_i A_i\in \mathcal{F}_n$.
\end{enumerate}  \\

Hence $\mathcal{F}_T$ is a $\sigma$-algebra.
\end{proof}


    \begin{lemma}
       If $S$ and $T$ are stopping times such that $S\leq T$, then $\mathcal{F}_S\subset\mathcal{F}_T$.
   \end{lemma}
\begin{proof}
For every $A\in\mathcal{F}_S$, we have
$$A\cap\{S\leq n\}\in\mathcal{F}_n, \forall n$$
Thus we have
$$A\cap\{T\leq n\}=\{A\cap\{S\leq n\}\}\cap\{T\leq n\}\in\mathcal{F}_n, \forall n$$
So $A\in\mathcal{F}_T$. This implies $\mathcal{F}_S\subset\mathcal{F}_T$.
\end{proof}


\begin{lemma}
  If $S$ and $T$ are \textbf{bounded} stopping times such that $S\leq T$ and $X=(X_n:n\geq 0)$ is a martingale, then $\mathbb{E}[X_T|\mathcal{F}_S]=X_S$,a.s.
\end{lemma}

\begin{proof}
As $S$ and $T$ are bounded, there exists $N\in\mathbb{N}$ such that $S\leq T\leq N$.

Firstly we prove that $\mathbb{E}[X_N\mathbb{1}_A]=\mathbb{E}\left[\mathbb{E}\left[X_N|\mathcal{F}_S\right]\mathbb{1}_A\right]$ for every $A\in\mathcal{F}_S$. By the definition of the conditional expectation, we have
$$\int_{A}X_NdP=\int_{A}\mathbb{E}[X_N|\mathcal{F}_S]dP$$

Hence
$$\mathbb{E}[X_N;A]=\mathbb{E}\left[\mathbb{E}\left[X_N|\mathcal{F}_S\right];A\right]$$

This implies $\mathbb{E}[X_N\mathbb{1}_A]=\mathbb{E}\left[\mathbb{E}\left[X_N|\mathcal{F}_S\right]\mathbb{1}_A\right]$.

For every $A\in\mathcal{F}_S$, we have
\begin{equation*}
    \begin{array}{rl}
    \mathbb{E}[X_N\mathbb{1}_A]
    =& \mathbb{E}\left[\mathbb{E}\left[X_N|\mathcal{F}_S\right]\mathbb{1}_A\right] \\
    =& \sum\limits_{i=1}^N \mathbb{E}\left[\mathbb{E}\left[X_N|\mathcal{F}_i\right]\mathbb{1}_A\mathbb{1}_{S=i}\right] \\
    =& \sum\limits_{i=1}^N \mathbb{E}\left[X_i\mathbb{1}_A\mathbb{1}_{S=i}\right] \\
    =& \mathbb{E}\left[X_S\mathbb{1}_A\right]
    \end{array}
\end{equation*}
The above equation implies that $\int_{A}X_NdP=\int_{A}X_SdP$ for every $A\in\mathcal{F}_S$. Thus $\mathbb{E}[X_N|\mathcal{F}_S]=X_S$. In the similar way, we have  $\mathbb{E}[X_N|\mathcal{F}_T]=X_T$. Thus we can conclude that
$$\mathbb{E}[X_T|\mathcal{F}_S]=\mathbb{E}\left[\mathbb{E}\left[X_N|\mathcal{F}_T\right]|\mathcal{F}_S\right]=\mathbb{E}[X_N|\mathcal{F}_S]=X_S$$

\end{proof}



\end{document}
